%\addbibresource{/home/jorgsk/phdproject/bibtex/jorgsk.bib}
The three topics that have been investigated in this thesis -- sequence
dependent abortive initiation, sequence dependent translation initiation, and
sequence dependent cleavage and polyadenylation -- all center around how the
genetic code regulates gene expression at these different levels.

When working with these projects we have made use of two distinct bioinformatic
approaches. For transcription and translation initiation we worked with
thermodynamic calculations in a gene-centric manner, while for the 3\ppp
cleavage and polyadenylation study we analyzed next generation sequencing data
to perform a genome-wide investigation. Having both these approaches in this
thesis illustrates well the methodological contrast that exists in biology
today, where ``-omic'' studies are increasingly finding applications in areas
that were previously only studied with traditional molecular biology
techniques.

What follows is a discussion of the way these two approaches have been used in
this thesis to provide new knowledge about gene expression. Additionally, the
two approaches will be compared in terms of how they facilitate different
research strategies and how the lead to different kinds of research challenges.
A summary of this discussion is given in Table \ref{tab:discussion_summary}.

\begin{table}[hb]
	\begin{center}
		\scalebox{0.8}{
		\begin{tabular}{l|ll}
			 & Genome-wide & Gene-centric \\
			\midrule
			Experiment type & RNA-seq & RT-PCR, western blot,\\
							&         & \textit{in vitro} transcription assay  \\
			 Data amount & 1 terabyte  & 1 megabyte \\
			 Data processing & Significant  & Minimal \\
			 Data interpretation & Challenging & Straight forward  \\
			 Experimental follow-up & No & Yes  \\
			 Study objective & General  & Specific  \\
			 Results & Overview, hypothesis  & Mechanistic insight \\
		\end{tabular}
		}
	\end{center}
	\caption{A summary of the discussion. The left column holds different
	characteristics that differ between the genome-wide and gene-centric
	studies.}
	\label{tab:discussion_summary}
\end{table}

\subsection{Data, research strategies, and challenges}
What fundamentally separates genome-wide and gene-centric studies in general
are the underlying data. The data from the transcription initiation study were
in terms of radioactive intensity of gel-migrating RNA oligonucleotides and was
generated \textit{in vitro}. This kind of data is straightforward to interpret
(abortive and full length RNA), and in the controlled, \textit{in vitro}
experimental set-up it is easy to perturb the input DNA sequence to achieve
variation in the output for hypothesis testing. The data from the work on
translation initiation were from western-blots and RT-PCR. Again the
interpretation of the data is relatively straightforward (relative protein and
mRNA concentrations).

Since the data from these gene-centric experiments were easy to interpret and
low in volume, they did not require extensive filtering or heavy computational
analysis. Importantly, this means that less time was devoted to data management
and data analysis, and more time was spent on interpretation of the biological
meaning of the experiments. In other words, for the gene-centric studies the
challenge was to understand the biological mechanism that was studied. For
translation initiation we needed to understand which steps during gene
expression are affected by mutations around the 5\ppp ribosome binding site.
For transcription initiation we had to combine how RNA polymerase moves
relative to DNA with abortive RNA synthesis; the challenge was to relate how
the equilibrium constant of translocation was linked with the abortive release
of short RNA.

On the other hand, the RNA-seq data underlying the study of 3\ppp cleavage and
polyadenylation imposed a different strategy for this project. Due to the size
and complexity of the data, the main challenge for the work was handling,
filtering, and processing the data. Much time was devoted to constructing the
analysis pipeline named Utail that was used for data filtering and analysis.
The datasets from all the different cell lines and compartments were over 1
terabyte in uncompressed form and took two days to process with a powerful,
multi-core workstation. As a direct consequence of the data-challenge, there
was less time devoted to the study of the biological mechanism. Perhaps
anticipating that this would be the case, the initial goal for the project was
made general: to characterize polyadenylation in different cell compartments
for different human cell lines. This is in contrast to the more detailed,
mechanism-oriented goals for the gene-centric experiments, namely the movement
of RNAP relative to DNA and the binding of the ribosome to mRNA.  Since less
time was spent on the computational aspect, the gene-centric approach
facilitated collaboration with wet-lab biologists about the biological problem
itself. The direct involvement of wet-lab biologists with the core scientific
problem is a likely reason why these studies resulted in follow up experiments.
Therefore, by having less computational focus, gene-centric approaches may
increase the chance of having follow-up experimental work, as they are more
likely to involve the collaboration of wet-lab biologists.

\subsection{Contribution to knowledge about gene expression}
Due to the differences in underlying data and study approach, the genome-wide
and gene-centric studies are bound to give rise to different types of
biological research questions that can be asked and answered. In chapter
\ref{chap:celB} we were able to point to ribosome binding as a limiting factor
for the heterologous expression of \textit{inf-$\alpha$2b}. This is a specific
type of information for a specific gene, but it does verify previous reports
that translation initiation can be a bottleneck for heterologous expression,
thereby adding to an existing body of evidence. The study also reinforced the
message that using RNA-RNA folding energy from secondary structures is a useful
approach to augments standard techniques in molecular biology like RT-PCR and
western blot when working with optimization of gene expression.

In chapter \ref{chap:initiation_paper} we solved the puzzle of why abortive
initiation is sequence dependent: it is because the initial transcribed
sequence affects RNAP's translocation bias, which we in turn linked to the
probability of backtracking and abortive RNA release. Even though this study
used only the N25 promoter, these results are likely to hold for all strong
promoters that undergo abortive initiation. Crucially, this study linked
several topics that had not been linked together before. First, it approximated
equilibrium constants of translocation from those of pyrophosphorolysis;
thereafter it linked translocation to the efficiency of promoter escape, before
finally completing the logical circuit by reasoning that backtracking from the
pre-translocated state is the rate limiting step during transcription
initiation. When we began the transcription initiation study we used the free
energies of the RNA-DNA and DNA-DNA duplexes to study transcription initiation.
These were at that time the known components that contribute energetically to
translocation by RNAP. However, it turned out that the free energy of the
interaction between the RNA 3\ppp dinucleotide and the RNAP active site was the
most important variable in the study. This is a good example of how one sets
out by building on the existing knowledge (the RNA-DNA hybrid and DNA-DNA
bubble contribute to RNAP translocation), and ends up adding a new component
(free energy associated with the 3\ppp dinucleotide matters more).

The main message from the study in chapter \ref{chap:polyA} is that
polyadenylation sites inferred from RNA-seq vary between different genomic
regions for different cellular compartments. In particular, we found that there
was a substantial increase in polyadenylation sites from poly(A)- RNA in the
intergenic regions of nuclear extract that was not present in the poly(A)+ RNA.
This may be an indication of regulation by gene expression through
degradation-related polyadenylation of intronic RNA. This contribution to the
understanding of gene expression is of a different kind than for the
gene-centric studies. Instead of analyzing a particular aspect of
polyadenylation, we looked broadly were able to identify general patterns. In
turn, this result should be used to investigate how intronic RNA is degraded in
mammalian cells. As such, the outcome of the genome-wide study becomes a
roadmap from which hypotheses can be formed.

To end the discussion, here follows a quote from a recent review by N.
Proudfoot called ``Ending the message: poly(A) signals then and now''
\cite{proudfoot_ending_2011}. Using early Sanger-like sequencing techniques in
1976, Proudfoot was the original discoverer of the AAUAAA hexamer, and deduced
both that this was the signal for polyadenylation and that the signal was
conserved across mammals \cite{proudfoot_3_1976}. In this regard, Proudfoot is
in a good position to comment on the recent surge in genome-wide
polyadenylation studies. He writes \cite{proudfoot_ending_2011}:

``It is abundantly clear that bioinformatic analysis of genomic data has
provided invaluable generality to our understanding of PAS [polyadenylation
signal] function in gene expression. However, current genome-wide analyses
often only provide bioinformatic correlations and lack direct functional
experimentation. Genomic analysis will only achieve its full potential when
bioinformatics can be matched by hypothesis-driven experimental approaches.''

%It is almost 33 years since abortive initiation was first discovered
%\cite{carpousis_cycling_1980}. After that, it took 6 years to document that the
%ITS affected promoter escape and abortive initiation
%\cite{kammerer_functional_1986}. Still after that, it took 10 more years for a
%comprehensive study to be published about how the sequence in the ITS affects
%the amount of full length and the amount of abortive product from a promoter
%\cite{hsu_initial_2006}. Further five years passed before the publication of
%the parameters for RNAP translocation that were used in this thesis to describe
%promoter escape efficiencies \cite{hein_rna_2011}.

%After these almost 33 years we have now shown that the \textit{in vitro}
%promoter escape efficiency for the \textit{E. coli}-infecting N25 phage
%$\sigma^{70}$-dependent promoter is strongly affected by the equilibrium of
%translocation during the initial transcription steps, presumably by affecting
%the probability of backtracking-induced abortive initiation. Now one must ask,
%how well does this translate to \textit{in vivo} conditions? What about the
%thousands of other $\sigma^{70}$ promoters? And what about the promoters of
%other $\sigma$ factors? What about other species that \textit{E. coli}? Does
%this also apply for eukaryotic cells, for which abortive initiation is also
%observed \cite{pal_initiationelongation_2003}? Perhaps 30 more years will pass
%before even the easiest of these questions are elucidated. Until that time,
%every integrative, complex model of gene expression in \textit{E. coli} must be
%made without a full understanding of transcription initiation and promoter
%escape.
