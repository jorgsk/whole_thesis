The work on this thesis has resulted in the main authorship by the thesis
author of one paper and the co-authorship of two papers. Below are listed the
papers (page number where they are found in this thesis in parenthesis) and the
authors and their contributions. For the papers where the thesis author has a
co-authorship, a paragraph describing his direct contribution is also given.
\\
\\
\textbf{Design and Optimization of Short DNA Sequences That Can Be Used as
5\protect\ppp Fusion Partners for High-Level Expression of Heterologous Genes
in \textit{Escherichia coli}} (page \pageref{vero_paper})\\
\\
\textbf{Veronika Kucharova} planned the work and experiments, performed
experiments, analysed data, and wrote the paper\\
\textbf{J\o rgen Skancke} performed data analysis; constructed
sequences for experimental testing; participated in the planning of experiments
for the sequenes he had produced; provided input to the writing of the
paper (30 \% of work)\\
\textbf{Trygve Brautaset} planned the work and wrote the paper\\
\textbf{Svein Valla} planned the work and wrote the paper\\
\\
The thesis authors' main contribution to this paper is the directed
computational method for screening for highly expressing 5\ppp variants. This
approach yielded consistently elevated transcript levels compared to the
original gene, showing that directed computational methods on the gene sequence
level can be successfully used to enhance transcript levels, which in this
study sidestepped the time-consuming step of generating and screening a
physical sequence library by random mutagenesis. The thesis author contributed
directly Figure 1 and the sequences tested for transcript and protein levels as
shown in Table 2.
\\
\\
\textbf{Thermodynamic modeling of initial transcription elucidates the
sequence dependence of promoter escape efficiency} (page \pageref{chap:initiation_paper})\\
\\
\textbf{J\o rgen Skancke} developed the idea and theory behind the paper;
performed all data analysis; developed the mathematical model, implemented the
model numerically, and performed all model simulations; used the model to
generate nucleotide sequences for experimental model validation; wrote the
paper and created all figures (90 \% of work) \\
\textbf{Nadav Bar} participated in discussions and gave input to writing\\
\textbf{Martin Kuiper} particiapted in discussions and gave input to writing\\
\textbf{Lilian M. Hsu} performed all wet lab experiments;
participated in discussions; wrote the paper\\
\\
\textbf{Landscape of transcription in human cells} (page \pageref{landscape})\\
\\
The list of authors can be read on page \pageref{landscape}. Due to the large
number of authors, only the contribution by the thesis author is listed below.
\\
\\
\textbf{J\o rgen Skancke} analysed data; provided results on polyadenylation
sites; gave input to the writing of the paper (5 \% of work)\\
\\
The thesis author's contribution to this paper is the outcome of a one year
stay at the biomedical research institution CRG (Center for Genomic
Regulation), where he contributed to a collaborative effort that led up to this
publication. The thesis author's direct contributions are the results on
polyadenylation sites given in the section ``Alternative transcription
initiation and termination''. His contribution complements the work that was
done by other authors on transcription start sites, showing what information
may be extracted from transcriptome datasets on the beginning and the end of
long RNAs. The methods and background study that led up to the thesis author's
published results are presented in chapter \ref{chap:polyA} on page
\pageref{chap:polyA}.
