%\addbibresource{/home/jorgsk/phdproject/bibtex/jorgsk.bib}
\section{Summary of the paper}
The primary aim of the publication by Kucharova et al.\, attached in the
Appendix on page \pageref{vero_paper}, was to express in \textit{E. coli} a
variant of the human interferon gene interferon-alpha called
\textit{inf-$\alpha$2b}. The protein product INF-$\alpha$2b is of
pharmaceutical interest as a dug for treating hepatitis C
\cite{manns_peginterferon_2001}. The expression of a foreign gene in a host
organism is called heterologous expression, and carries with it many challenges
\cite{gustafsson_codon_2004}. Since heterologous expression of INF-$\alpha$2b
is necessary to produce the drug in large quantities, and \textit{E. coli} is a
much used expression host, it is important to investigate the limitations and
mechanisms of \textit{inf-$\alpha$2b} expression in \textit{E. coli}.

Kucharova et al.\ reports that the \textit{inf-$\alpha$2b} gene is not
expressed in \textit{E.  coli} in its native form. A codon optimized version of
\textit{inf-$\alpha$2b} was obtained to investigate if the non-native codon
usage in the gene was behind the lack of expression. That is, the native human
codons were replaced with codons that are in high usage in \textit{E. coli}.
However, in spite of of codon optimization no detectable expression of
transcript or protein could be found. Protein product was only detected when a
5\ppp fusion tag was added to the \textit{inf-$\alpha$2b} gene. This indicated
that events occurring in the 5\ppp region of the gene is involved in regulating
the switch between expression and nonexpresssion of the \textit{inf-$\alpha$2b}
gene.

In their paper, Kucharova et al.\ suggest that translation initiation of the
\textit{inf-$\alpha$2b} transcript is a possible reason why the gene is not
expressed. Further, they note that codon usage in first translated codons are
also known to impact gene expression. That prompts them to investigate variants
of \textit{inf-$\alpha$2b} that make synonymous codon substitutions to reduce
RNA secondary structures around the ribosome binding site. Some of these
variants result in detectable transcript levels for \textit{inf-$\alpha$2b}.
However, the increase in transcript level is not followed by an increase in
protein level. This suggests that further barriers than ribosme binding lie
behind the poor expression of \textit{inf-$\alpha$2b}. Having established this,
Kucharova et al.\ move on to investigate the effect of different 5\ppp terminal
fusion peptides. They show that that several short versions of the celB fusion
peptide increase expression of the \textit{inf-$\alpha$2b} gene and several
other genes. Finally, they improve the expression level with the celB leader by
screening a random mutagenesis library of celB around the ribosome binding
site. In conclusion, these fusion peptides probably facilitate translation
initiation both by having favorable secondary structure and also by some
downstream effect that could not be achieved only by alleviating structures at
the ribosome binding site.

\section{My contribution to the paper}
My contribution to the paper was to perform the bioinformatic analysis, partake
in the planning of the experiments that involved changing codons and secondary
structure, and to assist in the writing process.

Initially, I performed RNA secondary structure analysis for a set of
celB-\textit{inf-$\alpha$2b} variants with variable expression levels (data not
shown in the paper). This analysis could not conclude that secondary structures
were the reason behind the variation in expression; specifically it could not
conclude that strong secondary structures were behind the lack of expression of
the unmodified \textit{inf-$\alpha$2b} gene.

Nevertheless, we proceeded to try to optimize the secondary structures around
the ribosome binding site (RBS) of the original \textit{inf-$\alpha$2b} gene
without celB, because this is an approach that has worked previously for
heterologous expression \cite{de_smit_secondary_1990}. For this work, I made a
combinatorial library of synonymous codons for the first nine codons in the
\textit{inf-$\alpha$2b} coding region. I labeled each of the resulting set of
over 8000 synonymous variants with an index for the rarity of their codons and
the folding energy of the secondary structure around the RBS. Some selected
variants from this library were chosen based on their folding energy and rarity
of codon usage.
