Gene expression -- the synthesis of RNA and protein -- requires most of the
cell's energy and is a highly regulated process at all its levels.  In this
thesis, three studies are presented which focus on the regulation of three
different levels of gene expression. One is centered on the regulation of
transcription initiation; the other is about translation initiation; while the
third looks at RNA post-transcriptional processing. The studies are united by
illustrating how the DNA sequence directly affects regulation at these
different levels. Where they stand out individually is that they rely on
different experimental data and require different theoretical techniques of
analysis.

The study on transcription initiation, which contributes the major impact of
this thesis, sheds light on a long-standing issue in the field of abortive
transcription initiation. It was known that the DNA sequence of the 20 first
transcribed basepairs could affect promoter escape properties; however it was
not known how this happened. We have shown, using free energy calculations,
that the manner in which RNA polymerase translocates in the first 15
transcribed basespairs explains the observed variation in promoter escape
efficiency. The study on translation initiation focuses on the problem of
heterologous expression of a human gene in a bacterial host. By changing
nucleotides in the first codons of the heterologous gene to reduce the free
energy potential of RNA secondary structures, we show that translation
initiation is a strong barrier for the expression of this gene. In the third
study, we analyze RNA-seq data to identify sites of cleavage and
polyadenylation in human transcripts in different cell lines in different cell
compartments. We find as expected an enrichment of polyadenylation sites which
have a close upstream polyadenylation signal AATAAA, but we also find an
unexpected enrichment of polyadenylation sites in intronic regions of the
nucleus.

Considering the contributions of this thesis for the understanding of gene
expression, we conclude that free energy calculations, when used in combiantion
with traditional molecular biology techniques, are efficient and useful tools
for investigating transcription and translation initiaiton.

Seen from a methodological point of view, the studies on transcription
initiation and translation were performed included collaboration with wet-lab
biology, while the study on polyadenylation were performed as an isolated
bioinformatical investigation. This affects the outcomes of the studies; for
the bioinformatical study no experimental validation of the results were
obtained, leaving the conclusion as a hypothesis. In conclusion, computational
biology is most meaningful when used in close collaboration with experimental
biology. 
