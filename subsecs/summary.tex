Gene expression -- the synthesis of RNA and protein -- requires most of the
cell's energy and is a highly regulated process at all its levels. In this
thesis, four studies are presented which focus on the regulation of three
different levels of gene expression. Two are centered on the regulation of
transcription initiation; the other on translation initiation; while the
third looks at RNA post-transcriptional processing. The studies are integrated
in the way that they illustrate how DNA sequences directly affect regulation at
these three different levels. The studies rely on different experimental data,
however, and require different theoretical methods of analysis.

The studies of transcription initiation focus on how the RNA polymerase (RNAP)
takes the step from promoter binding to promoter escape. Before reaching
promoter escape, RNAP may undergo abortive initiation, in which the nascent RNA
transcript is released from RNAP. At many promoters, RNAP undergoes repeated
abortive initiation, known as abortive cycling, before promoter escape is
reached. The DNA sequence composition of the 20 first transcribed basepairs has
been known to affect the efficiency of promoter escape, presumably by affecting the
probabilities of abortive initiation, but the mechanism for this effect was not
known. Using a thermodynamic model of transcription initiation, we have shown
that the manner in which RNAP translocates during initial
transcription explains the observed variation in promoter escape efficiency on
the N25 promoter. The key variable for linking translocation of RNAP during
initiation and promoter escape efficiency was the sequence of the RNA 3\ppp
dinucleotide, and not the free energy of the DNA bubble which had been
postulated by others. We proceed to verify our findings with follow-up
experiments, in which we used our thermodynamic model to construct N25 promoter
variants with predicted promoter escape efficiencies. The experiments agreed
well with the predictions, making a strong argument for translocation during
initial RNA synthesis as the major determinant of promoter escape efficiency.
While this study sheds light on  the sequence specificity of initial
transcription, it does not reveal the dynamics of the process. This therefore
is the focus on the second piece of work on initial transcription in this
thesis. In the work on kinetics, we combine two lines of separate experimental
evidence (single-molecule and bulk transcription studies) to identify the rate
constants for the key steps in initial transcription: the nucleotide addition
cycle, backtracking, and unscrunching and abortive RNA release. The most
important finding in this work was that the speed of pause-free transcription
during initiation is the same for initial transcription as reported for
transcription elongation. This tells us that scrunching and the expanded DNA
bubble do not seem to affect the kinetics of the combined steps of nucleotide
addition, translocation, and pyrophosphorolysis.

The study on translation initiation focuses on the problem of heterologous
expression of the human \textit{inf-$\alpha$2b} gene in \textit{E.\ coli}. A
known bottleneck for heterologous gene expression is the binding of the
ribosome to a folded translation initiation region on a messenger RNA.
Therefore, we introduced silent mutations in the first 9 codons of
\textit{inf-$\alpha$2b} that were predicted to reduce the free energy of RNA
secondary structures at the ribosome binding site to different degrees. This
approach produced some \textit{inf-$\alpha$2b} variants which showed an 
increased amount of \textit{inf-$\alpha$2b} transcript, indicating that
translation initiation is likely a strong barrier for the expression of this
gene in an \textit{E.\ coli} host.

In the fourth study, we focused on the process of 3\ppp cleavage and
polyadenylation of eukaryote RNAs. Cleavage and polyadenylation make up a part
of pre-RNA processing of many eukaryote RNAs, and is required for mRNA
stability and transport from the nucleus to the cytoplasm. To study 3\ppp
cleavage and polyadenylation, we analyzed RNA sequencing data to identify
polyadenylation sites in transcripts from different cell compartments across
12 human cell lines. We found over 160.000 polyadenylation sites, 80\% of
which were previously not annotated. In addition we found an unexpected
enrichment of polyadenylation sites in intronic regions of nuclear RNA. We
offer a discussion on how these sites may be signals of
polyadenylation-related nuclear degradation.

We have employed two different approaches in our investigation of the three
different aspects of gene regulation: a gene-centric approach for transcription
and translation initiation, and a genome-centric approach for 3\ppp cleavage
and polyadenylation. This has resulted in different types of research questions
being asked, different computational challenges, and different sorts of
results, with the gene-centric results being more mechanism oriented, and the
genome-centric being more general.

The main conclusion of the thesis as a whole is that kinetic modelling and
free energy calculations of RNA and DNA nucleotide chains have been
successfully applied in combination with traditional molecular biology
techniques; we base this on the results from our investigations of the
relationship between gene sequence and gene expression for RNAP-dependent
transcription initiation and ribosome-dependent translation initiation.
