%\addbibresource{/home/jorgsk/phdproject/bibtex/jorgsk.bib}
The target of this thesis was to use computational tools in combination with
wet-lab experiments to study regulation of gene expression.

In chapter \ref{chap:celB}, we found that decreasing the RNA secondary
structure around the ribosome binding site led to increased transcript
expression of the \textit{inf-$\alpha$2b} gene in \textit{E. coli}, likely as a
result of increased RNA stability through ribosome binding. We did this by
\textit{in silico} screening a library of variants of \textit{inf-$\alpha$2b}
that had synonymous mutations in the first 9 codons, and testing those variants
whose synonymous mutations led to decreased secondary structures in the
ribosome binding site. We conclude that heterologous gene expression of
\textit{inf-$\alpha$2b} in \textit{E. coli} may be limited by secondary
structures around the translation start site.

In chapter \ref{chap:initiation_paper}, we found that the sequence of the ITS
modulates the promoter escape efficiency through its effect on translocation.
This study proposes a solution for the 25-year case of why the initial
transcribed sequence modulates promoter strength and promoter escape
efficiency. In addition, contrary to what was previously assumed, we found no
role for the free energy of the scrunched DNA bubble in modulating promoter
escape efficiency. We conclude that Gibbs free energy of the DNA bubble is
unrelated to promoter escape efficiency, and that sequence dependent
productive yield from initial transcription can be explained by variation in
translocation caused by interactions between the 3\ppp dinucleotide and
internal sites in RNAP.

In chapter \ref{chap:kinetic_paper} we investigated the kinetics of initial
transcription on the N25 promoter using sequence-dependent abortive
probabilities. We conclude that the nucleotide addition cycle proceeds at the
same average speed for promoter-bound and promoter-free RNAP, and that the rate
limiting steps for promoter escape are unscrunching and abortive RNA release.

In chapter \ref{chap:polyA}, we classified evidence of polyadenylation in
different cell compartments for 12 human cell lines. We found evidence of
polyadenylation of poly(A)- RNA in intergenic and intronic regions. We
conclude that our findings indicate evidence of the recently discovered
nuclear degradation-related polyadenylation in human cells, and that RNA with
short poly(A) tails may be abundant in the nucleus.

Overall, the findings in this thesis have shed additional light on some of the
myriad ways in which gene sequences influence regulation of gene expression,
and shows how the potential of bioinformatics and computational biology is
best achieved when used predictively to inform experimental biology.
