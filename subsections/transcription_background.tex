%\bibliographystyle{plain}
%\bibliography{/home/jorgsk/phdproject/bibtex/jorgsk}
The molecular engine of transcription is the RNA polymerase (RNAP), which uses
nucleotide triphosphates (NTP) to synthesise an RNA molecule complementary to a
DNA template. Transcription can be divided into the stages of transcription
initiation, transciption elongation, and transcrition termination. 

\subsubsection{Transcription initiation}
RNA polymerase (RNAP) is by itself unable to initiate transcription without
being coupled to a $\sigma$ factor. The $\sigma$-RNAP complex is called the
holoenzyme and is capable recognizing sites along the bacterial DNA called
promoters in a $\sigma$-specific way. In \textit{E. coli} there are X sigma
factors, and each sigma factor controls the regulation of a particular gene
family. For example $\sigma^{70}$ recognizes promoters for housekeeping genes,
while $\sigma^{32}$ recognizes promoters for genes that are activated in case
of heat shock.

The promoter that is recognized by the holoenzyme is in general defined to be
the region from -80 to +20 relative to the transcription start site. This
region contains several sequence elements which have specificity for binding
with a specific $\sigma$ factor. Thus, promoters can be classified in terms of
which $\sigma$ factor they associate with. Most promoters are recognized by the
$\sigma^{70}$ factor. The most prominent binding elements in these promoters
are the -35 and -10 hexamers, but the AT rich UP element and the so called
discriminator region between -10 and the transcription start site (TSS) can
also regulate the binding affinity of $\sigma^{70}$ for the promoter.

Once the holoenzyme has bound to a promoter, the next stage of transcription
initiation sets in. First, RNAP melts the DNA from -11 to +2 relative to the
transcription start site. Now that DNA is double stranded, DNA is ready to be
used as a template, and RNAP begins joining together the first nucleotides to
from the nascent RNA. $\sigma^{70}$ however is still bound at the promoter, and
the $\sigma^{70}$-promoter bonds are not broken easily. In order to expose more
of the DNA template for pairing with the incoming nucleotides, RNAP melts
downstream DNA for each basepair that is incorporated. Since promoter bonds are
still not broken, RNAP at this time pulls DNA into itself in what has been
labeled 'scrunching'. It has been suggested that that buildup of stress from
the scrunching process eventually leads to the dissociation of sigma from the
promoter and allows RNAP to escape to the elongation phase of transcription.

\subsubsection{Transcription elongation}
After having achieved promoter escape, RNAP keeps incorporates incoming
nucleotides to the 5' end of the RNA molecule, thereby elongating it. The
location in RNAP where the incoming nucleotide is joined to the RNA is
called RNAP's active site. After a nucleotide has been incorporated, RNAP must
move one basepair downstream the DNA template to free up a space for the next
incoming substrate; this specific movement is called translocation. As RNAP
elongates, DNA is kept open in what is called the DNA bubble, which extends
roughly from -12 to +2 relative to the RNAP active site.

\subsubsection{Transcription termination}
Eventually, RNAP falls of DNA and releases its RNA product. Two distinct
mechanisms have been identified for the release of RNA from RNAP. In one, the Rho
protein binds a ribosome-free unstructured region of at least 97 nucleotides of
RNA and moves along RNA in the direction of the RNAP until they meet at RNAP
pause sites, and their interaction causes the release of both RNA and RNAP from
the DNA template \cite{ciampi_rho-dependent_2006}. The
Rho-independent, or intrinsic, mechanism of termination is through a strong
(GC-rich) RNA hairpin followed by a U-rich sequence on RNA. After the GC rich
sequence is free from the RNA exit channel it folds into a hairpin, and
RNAP proceeds to transcribe the U-rich sequence which gives a weak RNA-DNA
hybrid. Interactions between RNAP and the hairpin together with the weak hybrid
then causes the RNA to be released \cite{nudler_transcription_2002}.

In both cases, once RNA has been released, the affinity of RNAP to DNA is
greatly reduced and RNAP falls off. It is then free to associate with a
$\sigma$ factor and begin transcription again.

\subsubsection{Nucleotide incorporation and translocation}
Nucleotide incorporation happens in the post-translocated state of RNAP, where
the active site, or i+1, is free for basepairing with the incoming nucleoside
triphosphate (NTP). When the appropriate NTP has arrived through the secondary
channel and bound the DNA template in the active site, RNAP joins the incoming
substrate to the 3' end of the RNA with a phosphodiester bond, which releases
PPi. This puts RNAP in the pre-translocated state again, and RNAP must now
translocate once more to free up the active site for nucleotide synthesis. This
cycle of substrate binding, phosphodiester bond formation, and translocation,
constitute the forward motion of RNAP on DNA.

\subsubsection{Conceptual model for RNAP movement on DNA}
The movement of RNAP on DNA is thought to occur by a so called Brownian ratchet
mechanism. In this model, the backward movement of RNAP is structurally
prevented by the RNAP F bridge \cite{_ratchet_2005}, while the forward motion
is free to happen by brownian motion and as recently shown a translocation bias
\cite{hein_rna_2011}.

Even though the backward motion is prevented, it can occasionally happen,
leading to backtracked elongation complex. When this occurs, the 3' end of the
RNA product ends up in the RNAP secondary channel, blocking the active site
from binding incoming substrate. To rescue backtracked complexes, the GreA and
GreB transcription factors can bind in the secondary channel and stimulate the
intrinsic RNA cleavage activity of RNAP at the active site. When RNAP cleaves
the RNA, the backtracked RNA is end is released and a new 3' end is formed at
the active site, making RNAP again translocation-competent
\cite{toulme_grea_2000}.

The cause for the backward motion and other types of RNAP pausing during
transcription has been laid on the free energy balance between the RNA-DNA
hybrid, the DNA-DNA bubble, and an unknown stabilizing effect by
RNAP\cite{greive_thinking_2005}. By assuming the reaction steps that involve
each of these free energy values are in equilibrium
\cite{greive_thinking_2005}, the total free energy of the elongation complex
can be written as 
\begin{equation}
	\Delta G_{\text{Total}} = \Delta G_{\text{RNA-DNA}} + \Delta
	G_{\text{DNA-DNA}} + \Delta G_{\text{RNAP}}
	\label{eq:rnap_energy_balance}
\end{equation}
Depending on this energy term, it is proposed that RNAP will either move
forward, backtrack, pause, or even terminate.

\subsubsection{Computational models of transcription}
Since the first two terms of equation \eqref{eq:rnap_energy_balance} can be
calculated from published energy tables \cite{wu_temperature_2002}
\cite{santalucia_thermodynamics_2004}, several kinetic and thermodynamic models
of transcription elongation have been published
\cite{tadigotla_thermodynamic_2006-1} \cite{bai_sequence-dependent_2004}
\cite{guajardo_model_1997}. What they have in common is that in some form they
incorporate the terms from \eqref{eq:rnap_energy_balance} (in the case of
Tadigotla et. al the free energy from nascent RNA secondary structures close
to the RNA exit channel is used as well). When compared with measurement data,
however, these models leave much to be desired, showing that there is more to
RNAP processivity than the parameters in \eqref{eq:rnap_energy_balance}.

\subsubsection{A computational model of transcription initiation}
The difference between transcription initiation and transcription elongation is
i) that during initiation the RNA-DNA hybrid lacks its full length until +8/9,
ii) that the DNA-DNA bubble is scrunched and pulled into RNAP, and iii) that
RNAP is bound to $\sigma$ which again is bound to the promoter. The consequence
of these difference is that the TIC has a reaction pathway that doesn't exist
during elongation: abortive initiation.

Based on equation \eqref{eq:rnap_energy_balance} and the above differences, a
model for transcription initiation was previously published
\cite{xue_kinetic_2008}. This model attempts to cover explicitly both the
abortive process and the actual promoter escape event. Although these authors
manage to reproduce well the maximum size of abortive transcript and the
abortive to productive ration of the N25 promoter, they do not explain well the
abortive probabilities. This study investigates only the N25, N25anti and T7A1
promoters, even though it was published after a larger dataset was available
\cite{hsu_initial_2006}. Further, the T7A1 promoter varies not only in the ITS
sequence with the other two, but also in the core promoter sequence. Variation
in the core promoter has been shown to affect rates of abortive initiation
\cite{vo_vitro_2003-1}, yet these authors have not taken this variation into
account.
