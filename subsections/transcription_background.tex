%\bibliographystyle{plain}
%\bibliography{/home/jorgsk/phdproject/bibtex/jorgsk}
Transcription can be divided into the stages of transcription initiation,
transciption elongation, and transcrition termination.

\subsubsection{Transcription initiation}
The executor of transcription, the RNA polymerase (RNAP), is by itself unable
to initiate transcrioption without being coupled to a $\sigma$ factor. The
$\sigma$-RNAP complex is called the holoenzyme and is capable 
recognizing sites along the bacterial DNA called promoters in a
$\sigma$-specific way. In \textit{E. coli} there are X sigma factors, and each
sigma factor controls the regulation of a particular gene family. For example
$\sigma^{70}$ recognizes promoters for housekeeping genes, while $\sigma^{32}$
recognizes promoters for genes that are activated in case of heat shock.

The promoter that is recognized by the holoenzyme is in general defined to be
the region from -80 to +20 relative to the transcription start site. This
region contains several sequence elements which have specificity for binding
with a specific $\sigma$ factor. Thus, promoters can be classified in terms of
which $\sigma$ factor they associate with. Most promoters are recognized by the
$\sigma^{70}$ factor. The most prominent binding elements in these promoters
are the -35 and -10 hexamers, but the AT rich UP element and the so called
discriminator region between -10 and the transcription start site (TSS) can
also regulate the binding affinity of $\sigma^{70}$ for the promoter.

Once the holoenzyme has bound to a promoter, the next stage of transcription
initiation sets in. First, RNAP melts the DNA from -11 to +2 relative to the
transcription start site. Now that DNA is double stranded, DNA is ready to be
used as a template, and RNAP begins joining together the first nucleotides to
from the nascent RNA. $\sigma^{70}$ however is still bound at the promoter, and
the $\sigma^{70}$-promoter bonds are not broken easily. In order to expose more
of the DNA template for pairing with the incoming nucleotides, RNAP melts
downstream DNA for each basepair that is incorporated. Since promoter bonds are
still not broken, RNAP at this time pulls DNA into itself in what has been
labeled 'scrunching'. It has been suggested that that buildup of stress from
the scrunching process eventually leads to the dissociation of sigma from the
promoter and allows RNAP to escape to the elongation phase of transcription.

\subsubsection{Transcription elongation}
Afer having achieved promoter escape, RNAP keeps incorporates incoming
nucleotides to the 5' end of the RNA molecule, thereby elongating it. The
location in RNAP where the incoming nucleotide is joined to the the RNA is
called RNAP's active site. After a nucleotide has been incorporated, RNAP must
move one basepair downstream the DNA template to free up a space for the next
incoming substrate; this specific movement is called translocation. As RNAP
elongates, DNA is kept open in what is called the DNA bubble, which extends
roughly from -12 to +2 relative to the RNAP active site.

\subsubsection{Transcription termination}
Eventually, RNAP falls of DNA and releases its RNA product. Two distinct
mechanisms have been identified for the release of RNA from RNAP. In one, the Rho
protein binds a ribosome-free unstructured region of at least 97 nuceltodes of
RNA and moves along RNA in the direction of the RNAP until they meet at RNAP
pause sites, and their interaction causes the release of both RNA and RNAP from
the DNA template \cite{ciampi_rho-dependent_2006}6. The
Rho-independent, or intrinsic, mechanism of termination is through a strong
(GC-rich) RNA hairpin followed by a U-rich sequence on RNA. After the strong
hairpin has formed, RNAP is proceeds to transcribe the U-rich sequence.
Interactions between RNAP and the hairpin, the U-rich sequence in the RNA-DNA
hybrid, and other part of RNA, then causes the RNA to be released
\cite{nudler_transcription_2002}.

In both cases, once RNA has been released the affinity of RNAP to DNA is
greatly reduced and RNAP falls off. It is then free to associate with a
$\sigma$ factor and begin transcription again.

\subsubsection{Mechanism of RNAP movement}
Several models were originally put forth for the movement of RNAP on DNA, but
recent single-molecule studies indicate that RNAP most likely moves by a
thermal ratchet mechanism.

\subsubsection{Nucleotide incorporation}
Focus on this part here, so you don't have to mention it here and there in the
other sections. Also, this will provide some background for the models.

\subsubsection{Transcription pausing}

\subsubsection{Transcription backtracking}
