%\bibliography{/home/jorgsk/phdproject/bibtex/jorgsk}
Transcription is the synthesis of an RNA molecule from a DNA template. The
macromolecule that performs transcription is the DNA-dependent RNA polymerase
(RNAP). In bacteria, RNAP consists of the subunits $\beta$, $\beta^{\prime}$,
$\omega$, and two $\alpha$ subunits and has a molecular mass of around 400 kDa.
The bacterial polymerase is conserved in structure and function with those
found in in archea and eukaryotes \cite{borukhov_rna_2008}, showing that the
mechanism of RNA synthesis is virtually unchanged in all cell types.

Here, we will review the role of RNAP in initiation, elongation, and
termination of transcription in bacteria. Special care will be given to the
topic of abortive transcription initiation and how RNAP moves along DNA.

\subsubsection{Promoter localization}
Before RNAP can start transcription it must localize a transcription start site
(TSS) on DNA which mark the beginning of genes. These sites are advertised by
promoter elements, which typically stretch from around -80 to +20 relative to
the TSS, where - indicates downstream (5\p) and + indicates upstream (3\p).
RNAP cannot bind to the promoters directly, but must first associate with a
regulatory protein called the $\sigma$ ($\sigma$) factor; the $\sigma$ factor in turn
only binds strongly to promoters when together with RNAP \cite{paget_70_2003}.
In \textit{E. coli} there are seven types of $\sigma$ factors, where each $\sigma$
factor regulates the binding of RNAP to a particular gene family
\cite{osterberg_regulation_2011}. For example, $\sigma^{70}$ recognizes
promoters for housekeeping genes, while $\sigma^{32}$ recognizes promoters for
genes that are activated in case of heat shock. Most promoters are of the
$\sigma^{70}$ type, and we will here focus on this type of promoters. The most
prominent $\sigma$-binding elements in these promoters are the -35 and -10
hexamer (six-nucleotide) motifs, with the consensus sequences TTGACA (-35) and
TATAAT (-10), see Figure X adopted from \cite{haugen_fine_2008}. The major
determinants for how strongly $\sigma^{70}$ binds to a promoter are how much
the -35 and -10 motifs vary from the consensus and the distance in nucleotides
between these elements. Other sequences that affect RNAP-$\sigma^{70}$ binding
are an AT rich element upstream -35 \cite{ross_third_1993}, the extended -10
element, and the discriminator sequence between the -10 element and the TSS
\cite{haugen_fine_2008}, see Figure X. In general, the stronger $\sigma^{70}$
binds to a promoter the higher the downstream gene will be expressed.

\subsubsection{Transcription initiation}
% nucleotide = base-sugar-phosphate (1 or 3)
% nucleoside = base-sugar
% BUT! this confusing terminology exists: ``nucleotide with 3 phosphates'' =
% nucleoside triphosphate.
Once the RNAP-$\sigma$ complex has bound to a promoter, transcription
initiation may commence. First, $\sigma$ mediates the opening of the double
stranded DNA from -11 to +2 relative to the TSS. The single stranded DNA around
RNAP is referred to as the DNA bubble. In the DNA bubble, bases are exposed and
ready to be used as a template. RNAP is positioned so that the nucleotide at
the TSS is at RNAP's active site (the active site is the name for the part of
RNAP where RNA synthesis occurs). At the active site, the +1 nucleotide is free
to base-pair with a matching nucleoside triphosphates (NTP) which reaches the
active site in RNAP from the outside through diffusion through the NTP entry
channel.  After base-pairing, the +1 DNA nucleotide and the NTP form the first
basepair of the RNA-DNA hybrid. In order to incorporate the second nucleotide,
RNAP must first undergo translocation. Translocation is a one-basepair movement
step on DNA which is necessary to place the active site in the +2 position.
After an NTP has bound the +2 DNA nucleotide, RNAP catalyzes a bond between the
+1 NTP and the +2 NTP, a reaction which releases pyrophosphate (PPi). This
catalysis forms the first two bases of RNA. RNA synthesis continues following
the steps of translocation, NTP binding, and catalysis. At each translocation
step, one basepair of the downstream DNA bubble is opened up, and the nascent
RNA is moved toward the RNA exit channel.  See Figure X for an illustration of
RNAP during transcription initiation.

This cycle of translocation and nucleotide incorporation goes on although $\sigma$
is still bound to the promoter. As a consequence, the polymerase is not free to
move downstream during translocation, which normally happens during
transcription elongation. Instead, RNAP pulls DNA into itself in a process
that has been labeled ``scrunching''. It has been suggested that buildup of
free energy from the scrunched DNA-bubble eventually leads to the dissociation
of sigma from the promoter which allows RNAP to escape the promoter and reach
the elongation phase of transcription \cite{revyakin_abortive_2006}.

\subsubsection{Abortive initiation}
Promoter escape is only one out of two possible outcomes for a transcription
initiation attempt. The other possibility is that the initiation attempt fails;
then the nascent RNA is released and RNAP must begin synthesis \textit{de novo}
at the TSS. The term for this is abortive initiation. If RNAP attempts
promoter escape several times, this is called abortive cycling. Most initiation
complexes (RNAP, $\sigma$, and DNA) will abort if the promoter-$\sigma$
contacts are not broken by around the +16 position
\cite{lilian_m_promoter_2002}. The aborted RNA snippets are generally no longer
than 15 nucleotides, but lengths up to 20 have been observed
\cite{chander_alternate_2007}. In \textit{in vitro} experiments using
\textit{E. coli} extracts, abortive initiation is the rule rather than the
exception. Some promoters have an \textit{in vitro} ratio of aborted to
successful transcription initiation attempts of more than 300
\cite{hsu_initial_2006}. This indicates that RNAP may on average spend
considerable time in abortive cycling before achieving promoter escape.

When comparing abortive initiation at different promoters, it became clear that
both the lengths of the abortive transcripts and the extent of abortive cycling
were promoter-specific \cite{hsu_vitro_2003}. It was for example shown by
mutagenesis that the sequence in the core promoter region directly affects the
rate of promoter escape \cite{vo_vitro_2003}. But the most striking observation
was that the promoter sequence after +1 had a strong effect on the lengths of
the abortive transcripts \cite{hsu_vitro_2003}. Hsu et al. investigated the
abortive properties of a comprehensive library of 43 promoter variants with a
randomized +1 to +20 initial transcribed sequence (ITS). They showed that
sequence variation in the ITS could result in a 20-fold difference in the rate
of productive (as opposed to abortive) promoter escape \cite{hsu_initial_2006}.
This shows that steps during early RNA synthesis affect whether transcription
initiation is successful or not.

For a while it was speculated that abortive initiation was an \textit{in vitro}
artefact. However, recently small abortive RNAs have been identified \textit{in
vivo} \cite{goldman_direct_2009}. Following the discovery that abortive
transcripts do appear \textit{in vivo}, it was not certain whether the abortive
transcripts had any cellular function or if they were merely artefacts of the
transcription initiation process. Yet in the last two years two studies have
shown two different functions of these short transcripts. In one, aborted
transcripts from the $\phi$10 promoter were found to deactivate a
transcriptional terminator hairpin \cite{lee_tiny_2010}. In the other, short
abortive products of 2 to 4 nucleotides in length were found to act as primers
for the RNA polymerase \textit{in vivo} \cite{goldman_nanornas_2011};
previously, it was not known if RNAP, unlike the DNA polymerase, could use
primers \textit{in vivo}.

It is still not clear if abortive initiation is rate-limiting for natural
promoters \textit{in vivo}. However, \textit{in vitro} and \textit{in vivo}
experiments in \textit{E. coli} have shown that the transcription factors GreA
and GreB greatly reduce the amount of abortive initiation from the
N25$_{\text{anti}}$ promoter \cite{hsu_escherichia_1995}, suggesting that these
transcription factors may reduce any promoter-specific variability in abortive
initiation in the cell.

GreA and GreB act by stimulating the cleavage of RNA in the active site of
backtracked RNAP \cite{toulme_grea_2000}. Backtracked RNAP have the 3\p end of
the nascent RNA protruding out the NTP entry channel. This puts the RNA 3\p end
out of alignment with the active site which blocks further RNA synthesis.
GreA and GreB can rescue backtracked RNAP by cleaving RNA in the active site,
thereby realigning the RNA 3\p end at the active site. It is thus possible that
initially transcribing complexes on the path to abortive initiation \textit{in
vivo} are frequently targeted by GreA/B and thus avoid extensive abortive
cycling. In support of an \textit{in vivo} role for GreA in mitigating
abortive initiation, it was found that GreA resolves promoter-proximal stalling
of RNAP \cite{kusuya_transcription_2011}. However, more work is still needed
to confirm the precise role GreA/B play in abortive initiation \textit{in
vivo}.

\subsubsection{Transcription elongation}
Once the $\sigma$-promoter bonds are broken, RNAP is free to elongate onto the
downstream DNA. However, even though $\sigma$ has broken contacts with the
promoter, it is still in a complex with RNAP which has now initiated
elongation. There are some steps during early transcription where the nascent
RNA is thought to weaken the link between RNAP and $\sigma$. After the nascent
transcript reaches a length of more than 15 nucleotides, it should have
physically displaced parts of $\sigma^{70}$ \cite{mekler_structural_2002,
nickels_interaction_2005} in order reach the RNA exit channel. It is thought
that $\sigma$ is released stochastically from this weakened complex, as
sometimes $\sigma$ has been found retained with RNAP in far downstream sequences
\cite{mooney_sigma_2005}. Several studies have found that as a consequence of
$\sigma^{70}$ not dissociating, the still-attached sigma can rebind
promoter-like elements during transcription elongation, causing RNAP to pause
in its track \cite{ring_function_1996, kapanidis_retention_2005,
raffaelle_holoenzyme_2005}.

Transcription elongation happens with great processivity: RNAP can accurately
transcribe tens of thousands of nucleotides without dissociating from DNA. In
spite of this processivity, elongation does not happen at an even pace. RNAP
will reproducibly pause or backtrack at certain sites
\cite{herbert_sequence-resolved_2006}. Sometimes this pausing or backtracking
has a regulatory function, for example to allow time for proper folding of the
nascent RNA \cite{landick_r_regulatory_2006}. Thus transcription elongation is
not just a mandatory step in going from gene to RNA, but yet another stage of
transcription where regulation may take place.

\subsubsection{Transcription termination}
Eventually, RNAP will dissociate DNA and release its RNA product. Two distinct
mechanisms have been identified for the release of RNA from RNAP. In one, the
protein Rho binds an unstructured region of the nascent RNA and moves along RNA
in the direction of RNAP until they meet at RNAP pause sites. At these pause
sites the interaction between Rho and RNAP causes the release of both RNA and
RNAP from the DNA template \cite{ciampi_rho-dependent_2006}. The
Rho-independent mechanism of termination begins by the formation of a strong
(GC-rich) RNA hairpin on the nascent RNA right outside the RNA exit channel. If
this hairpin is followed by a T-rich sequence on DNA which destabilizes the
RNA-DNA hybrid, interactions between the hairpin and RNAP cause RNAP to release
its hold on the RNA. However, the details of the process are not clear
\cite{nudler_transcription_2002}.

In both cases, once RNA has been released, the affinity of RNAP to DNA is
greatly reduced and RNAP itself disengages DNA. It is then free to associate
with a $\sigma$ factor and begin transcription anew.
