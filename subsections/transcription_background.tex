%\bibliography{/home/jorgsk/phdproject/bibtex/jorgsk}
Transcription is the synthesis of an RNA molecule from a DNA template, and the
macromolecule that performs transcription is the DNA-dependent RNA polymerase
(RNAP). RNAP consists of the $\beta$, $\beta^{\prime}$, $\omega$, and two
$\alpha$ subunits, has a molecular mass of around 400 kDa, and is conserved
in structure and function across all kingdoms of life \cite{borukhov_rna_2008}.

Here, we will review the role of RNAP in initiation, elongation, and
termination of transcription. Special care will be given to the topic of
abortive transcription initiation and the model for how RNAP moves along DNA.

\subsubsection{Promoter localization}
Before RNAP can start transcription it must localize a transcription start site
(TSS). These sites are advertised by promoter elements, which stretch from -80
to +20 relative to the TSS. RNAP cannot bind to the promoters directly, but
must first associate with a regulatory protein called the sigma ($\sigma$)
factor; the sigma factor in turn only binds strongly to promoters when together
with RNAP \cite{paget_70_2003}. In \textit{E. coli} there are seven types of
sigma factors, and each sigma factor controls the binding to a particular gene
family \cite{osterberg_regulation_2011}. For example $\sigma^{70}$ recognizes
promoters for housekeeping genes, while $\sigma^{32}$ recognizes promoters for
genes that are activated in case of heat shock. Most promoters are of the
$\sigma^{70}$ type, and we will here focus on this type of promoters. The most
prominent $\sigma$-binding elements in these promoters are the -35 and -10
hexamers, but an AT rich UP element and the so called discriminator region
between -10 and the (TSS) can also affect the binding affinity of $\sigma^{70}$
for the promoter \cite{ross_third_1993, haugen_fine_2008}.

\subsubsection{Transcription initiation}
% nucleotide = base-sugar-phosphate (1 or 3)
% nucleoside = base-sugar
% BUT! this confusing terminology exists: ``nucleotide with 3 phosphates'' =
% nucleoside triphosphate.
Once the RNAP-$\sigma$ complex has bound to a promoter, transcription
initiation may commence. First, sigma mediates the melting of the double
stranded DNA from -11 to +2 relative to the transcription start site. Now that
it is single stranded, DNA is ready to be used as a template. RNAP will
position itself so that free DNA base of the transcription start site (+1) will
be in the active site of the polymerase. Now, this DNA base is free to
base-pair with an incoming nucleoside triphosphate (NTP), forming the first RNA
nucleotide. In order to incorporate the second nucleotide, the active site must
be cleared of the first nucleotide and be positioned in the +2 position. This
movement for the purpose of incorporating a new nucleotide is called
translocation. Once RNAP has translocated, the nucleotide corresponding to the
+2 DNA base can bind. To join the two nucleotides, RNAP catalyzes a
phosphodiester bond from the phosphate group of the +2 nucleotide to the ribose
sugar of the +1 nucleotide, a reaction which releases pyrophosphate (PPi).
Subsequently RNAP must translocate again to make space for the +3 nucleotide.
At each translocation step, one basepair of the downstream DNA bubble must
open, and the nascent RNA is moved toward the RNA exit channel
\cite{o_maoileidigh_unified_2011}.

This cycle of translocation and nucleotide incorporation goes on although sigma
is still bound to the promoter. This has the consequence that instead of being
free to move downstream with each translocation step to expose more DNA
template, RNAP must pull DNA into itself in what has been labeled
``scrunching''.  It has been suggested that buildup of stress from the
scrunching process eventually leads to the dissociation of sigma from the
promoter and allows RNAP to escape the promoter to the elongation phase of
transcription \cite{revyakin_abortive_2006}.

\subsubsection{Abortive initiation}
Promoter escape is only one out of two possible outcomes for a transcription
initiation attempt. If the promoter-$\sigma$ contacts are not broken by around
the +16 position, most initiation complexes will have aborted the escape
attempt and released the nascent RNA into solution
\cite{lilian_m_promoter_2002}. The aborted transcripts are generally no longer
than 15 nucleotides, but lengths up to 20 have been observed
\cite{chander_alternate_2007}. In \textit{in vitro} experiments, abortive
initiation is the rule rather than the exception, with some promoter variants
having a ratio of aborted to successful transcription initiation attempts of
more than 300 \cite{hsu_initial_2006}. This indicates that RNAP may on average
spend considerable time in abortive cycling before achieving promoter escape.

When comparing abortive initiation at different promoters, it became clear that
both the lengths of the abortive transcripts and the extent of abortive cycling
were promoter-specific \cite{hsu_vitro_2003}. Although elements in the core
promoter region affected abortive initiation \cite{vo_vitro_2003}, the most
striking observation was that the sequence composition of the +1 to +20
initially transcribed sequence (ITS) had a strong effect on the lengths of the
abortive transcripts \cite{hsu_vitro_2003}. By investigating the abortive
properties of a comprehensive library of 43 random ITS variants with the N25
core promoter it was later shown that the variation in ITS composition could
result in a 20-fold difference in the rate of productive (as opposed to
abortive) synthesis \cite{hsu_initial_2006}. 

For a while it was speculated that abortive initiation was an \textit{in vitro}
phenomenon. However, recently small abortive RNAs have been identified
\textit{in vivo} \cite{goldman_direct_2009}. It was also for a long time not
know if the short abortive transcripts had any cellular function or if they
were merely artefacts of the transcription initiation process. Yet in the last
two years two studies have shown two different uses of these short transcripts.
In one, aborted transcripts from the $\phi$10 promoter were found to deactivate
the a transcriptional terminator hairpin \cite{lee_tiny_2010}. In the other,
short abortive products of 2 to 4 nucleotides in length were found to act as
primers for the RNA polymerase \textit{in vivo} \cite{goldman_nanornas_2011};
previously, it was not known if RNAP, unlike the DNA polymerase, could use
primers \textit{in vivo}.

So far, it has not been shown if abortive initiation is rate-limiting for
natural promoters \textit{in vivo}. However, \textit{in vitro} and \textit{in
vivo} experiments have shown that the transcription factors GreA and GreB
greatly reduce the amount of abortive initiation from the N25$_{\text{anti}}$
promoter \cite{hsu_escherichia_1995}, indicating that these transcription
factors could act in the cell to reduce any promoter-specific variability in
abortive initiation. GreA and GreB act by stimulating the cleavage of RNA in
the active site of backtracked RNAP \cite{toulme_grea_2000}. Backtracked RNAP
have the 3\p end of the nascent RNA protruding out the NTP entry channel, thus
putting the RNA out of alignment with the active site and blocking further RNA
synthesis. GreA and GreB can rescue backtracked RNA by cleaving RNA in the
active site, thereby releasing the 3\p end of RNA and freeing up the active
site for binding to incoming NTP. It is thus possible that initially
transcribing complexes on the path to abortive initiation \textit{in vivo} are
frequently targeted by GreA/B and thus avoid extensive abortive cycling.  In
support of an \textit{in vivo} role for GreA in mitigating abortive initiation,
it was found that GreA resolves promoter-proximal stalling of RNAP
\cite{kusuya_transcription_2011}. However, more work is still needed to confirm
the precise role GreA/B play in abortive initiation \textit{in vivo}.

\subsubsection{Transcription elongation}
Once the sigma-promoter bonds are broken, RNAP is free to elongate onto the
downstream DNA. However, even though sigma has broken contacts with the
promoter, it is still in a complex with RNAP which has now initiated
elongation. There are some steps during early transcription that are predicted
to weaken the link between RNAP and sigma. After the nascent transcript reaches
a length of more than 15 nucleotides, it should have displaced both the R3.2
linker and region 4 of $\sigma^{70}$ \cite{mekler_structural_2002,
nickels_interaction_2005} in order to exit RNAP. It is thought that sigma is
released stochastically from this weakened complex, as sometimes sigma has been
found retained to RNAP in far downstream sequences \cite{mooney_sigma_2005}.
Several studies have found that as a consequence of this, the still-attached
sigma can rebind promoter-like elements during transcription elongation,
causing RNAP to pause in its track \cite{ring_function_1996, kapanidis_retention_2005,
raffaelle_holoenzyme_2005}.

Transcription elongation happens with great processivity: RNAP can accurately
transcribe tens of thousands of nucleotides without dissociating from DNA. This
stability has been attributed to the $\sim$ 9 nt RNA-DNA hybrid
\cite{nudler_rna-dna_1997}. In spite of this processivity, elongation does not
happen at an even pace. RNAP will reproducibly pause or backtrack at certain
sites \cite{herbert_sequence-resolved_2006}, and sometimes this pausing or
backtracking has a regulatory function, for example to allow time for proper
RNA folding of the nascent RNA \cite{landick_r_regulatory_2006}.

\subsubsection{Transcription termination}
Eventually, RNAP will dissociate DNA and release its RNA product, or possibly
the other way around. Two distinct mechanisms have been identified for the
release of RNA from RNAP. In one, the Rho protein binds an unstructured region
of the nascent RNA and moves along RNA in the direction of RNAP until they meet
at RNAP pause sites. At these pause sites the interaction between Rho and RNAP
causes the release of both RNA and RNAP from the DNA template
\cite{ciampi_rho-dependent_2006}. The other, Rho-independent mechanism of
termination is by the formation of a strong (GC-rich) RNA hairpin on the
nascent RNA right outside the RNA exit channel. If this hairpin is followed by
a T-rich sequence on DNA which destabilizes the RNA-DNA hybrid, interactions
between the hairpin and RNAP cause RNAP to release its hold on the RNA,
although the details of the process are not clear
\cite{nudler_transcription_2002}.

In both cases, once RNA has been released, the affinity of RNAP to DNA is
greatly reduced and RNAP itself disengages DNA. It is then free to associate
with a sigma factor and begin transcription anew.
