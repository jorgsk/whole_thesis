%\bibliography{/home/jorgsk/phdproject/bibtex/jorgsk}

The underlying mechanism of cellular life is a closed loop. To proliferate, the
cell must duplicate its DNA. To duplicate DNA, the cell depends on proteins and
enzymatic RNA for energy uptake and DNA synthesis, and those very proteins and
RNAs must be produced from DNA templates. The loop then goes from DNA to
protein and RNA back to DNA again.

In this work we will focus on the part of the loop that goes from DNA to
protein and enzymatic RNA. This step is often reffered to as gene expression,
since each protein and RNA is the result of transfer of information --
expression -- of a gene.

Gene expression can be mechanistically separated into many stages. The exact
number of stages depends on the point of view that is relevant, but the main
steps are clear: gene expression takes place through the steps of transcription
and translation. First, a section of a DNA molecule is transcribed into an RNA
molecule. Then this RNA molecule is translated into a protein.

A basic property of celluar life is that gene expression is carefully
regulated. The bacterial cell \textit{Eschericia coli} contains roughly 4000
genes, but at a given time only a subset of these are expressed. The choice of
which genes to express at a given time depends on both extracellular and
intracellular input to the genetic program.

In this work, we study several cases of regulation of gene expression. While
much of the global orchestration of expression is done by DNA and RNA binding
transcription factors, we will here instead study specific parts of the gene
sequence that affects expression in ways that are unrelated to transcription
factors. The two phases of gene expression that will be studied are the
initiation stages of both transcription and translation, and we will see how
for a given gene the same DNA sequence can sometimes control both these events.
