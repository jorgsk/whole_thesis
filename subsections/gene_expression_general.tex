%\bibliography{/home/jorgsk/phdproject/bibtex/jorgsk}

The cell is considered to be the unit of life. A single celled organism manages
by itself to absorb nutrients from its environment and use those nutrients to
navigate itself and to replicate itself. On the surface, life is then about
consuming nutrients and replicating. It is possible to stop with this
understanding of life and move on. But if we want to understand how the cell
uses these external metabolites like sugars and fatty acids to drive cell
division and cellular locomotion -- and we do -- then we must look under the
surface to the cell interior. Inside the cell we see a crowded milieu of DNA,
RNA, and protein molecules in constant collision and reaction. But we also see
a system to the chaos, and that system begins with DNA being the blueprint or
the instruction set from which all cellular programs are executed. Ultimately,
the DNA blueprint is used to produce RNA and protein molecules that have
mechanical, structural, or regulatory roles in the cell, such as cell division
or motility.

In this thesis we shall be concerned with the production of RNA and protein
from DNA. Overall, this process is called gene expression, because genes are
those regions of DNA which contain the information passed on to RNA and protein.

Gene expression can be mechanistically separated into many stages. The exact
number of stages depends on the point of view that is relevant, but the two
main stages will always be transcription and translation. In transcription, a
gene is processed to produce an RNA copy. In translation, that RNA copy is
processed to produce a protein. Sometimes, gene expression stops after
transcription, when the RNA is not a protein template but an enzyme on its own.
In this work, we will mostly deal with RNA that is translated, so called
messenger RNA (mRNA).

A basic property of gene expression is that it is carefully regulated. The
bacterial cell \textit{Eschericia coli} contains roughly 4000 genes, but at a
given time only a subset of these are expressed. The choice of which genes to
express at a given time depends on the environmental condition the cell finds
itself in.

In this work, we study several cases of regulation of gene expression.
Specifically, we will study the three phases of gene expression that
corresponds to the i) initiation of transcription, ii) the termination of
transcription, and iii) the initiation of translation. The work on
transcription and translation initiation has been done on bacterial cells,
while the work on transcription termination has been done on human cells.
