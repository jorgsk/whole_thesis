%\bibliography{/home/jorgsk/phdproject/bibtex/jorgsk}

The cell is the unit of life. A single celled organism manages can absorb
nutrients from its environment and will use those nutrients to navigate,
replicate and maintain homeostasis. From observing the cell from the outside,
life seems to be about consuming nutrients and replicating. But if we want to
understand how the cell uses these external metabolites like sugars and fatty
acids to drive cell division and cellular locomotion -- and we do -- then we
must look under the surface to the cell interior. Inside the cell we see a
crowded milieu of DNA, RNA, and protein molecules in constant collision and
reaction. But we also see a system to the chaos, and that system begins with
DNA being the blueprint or instruction set from which all cellular programs are
executed.  Ultimately, the DNA blueprint is used to produce RNA and protein
molecules that through interactions on the micro scale are able to cause macro
scale changes such as cell division or motility.

Genes are the parts of DNA molecules which are copied into RNA, and gene
expression is another term for RNA and protein production. Gene expression
consist broadly of two stages: transcription, in which an RNA copy is made from
DNA, and translation, in which a protein is created based on the genetic code
in the RNA sequence. For some genes, for example genes for transfer RNA (tRNA)
and ribosomal RNA (rRNA), gene expression is confined to transcription alone.
Here, the RNA molecule is on its own competent to perform a structural or
enzymatic function or joins with other RNA and protein in complexes. In this
work however, we will deal with RNA that is translated to protein, so called
messenger RNA (mRNA).

A basic property of gene expression is that it is carefully regulated. The
bacteria \textit{Eschericia coli} contains more than 4000 protein coding genes
in its genome, but at a given time only a subset of these are expressed. Which
genes are expressed at a given time depends on the environmental condition the
cell finds itself in. Examples of environmental condigions are nutrient
availability, temperature, salt concentrations, signals from other cells, and
toxins; each of these input signals will have unique or partially overlapping
gene expression responses honed over the course of evolution.

In this work we will not study how any of the mentioned environmental
conditions affect the pattern of gene expression. Instead, we will study gene
expression itself at several different stages. Specifically, we study i) the
initiation of transcription, ii) the termination of transcription, and iii) the
initiation of translation. The work on transcription and translation initiation
has been done on \textit{E. coli} (bacterial) cells, while the work on
transcription termination has been done with human (eukaryotic) cells. It is
estimated that bacterial and eukaryotic cells have evolved independently for
over two billion years \cite{vellai_origin_1999}. Yet throughout that time, the
key players in this thesis, the RNA polymerase and the ribosome, have remained
structurally and functionally conserved between those cell types.

In the following chapter we will review those stages of transcription and
translation that are relevant for the work at hand. The reviews will cover both
the biological and computational/theoretical background needed to read the
thesis as a whole.
