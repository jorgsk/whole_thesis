%\bibliography{/home/jorgsk/phdproject/bibtex/jorgsk}
A single celled organism can absorb nutrients from its environment and can use
those nutrients to navigate, replicate and maintain homeostasis. Therefore many
have defined the cell as the unit of life.  To understand how life works, we
must look under the surface of the cell, to the cell interior. Inside the cell
we see a crowded milieu of DNA, RNA, and protein molecules in constant
collision and reaction. A human spectator to the cell's inside would see a blur
of molecules racing here and there and little apparent order. But there is an
order to the cell, and that order begins with DNA being the instruction set
from which all cellular programs are executed. The DNA blueprint is used to
produce RNA and protein molecules that through interactions on the micro scale
are able to cause macro scale changes such as cell division or motility.

The production of RNA and protein molecules from DNA is called gene expression,
and will be the theme that unites the work in this thesis. Gene expression
consists of two stages: transcription, in which an RNA copy of the gene is made
from DNA, and translation, in which a protein is created using the genetic
information in RNA sequence. For some genes, for example genes for transfer RNA
(tRNA) or microRNA, gene expression is confined to transcription alone. Here,
the RNA molecule is on its own competent to perform a structural or enzymatic
function, or it may join with other RNAs or proteins in complexes that perform
those functions. Most genes however encode RNA that will be translated into
protein. These RNAs are called messenger RNA (mRNA). The work in this thesis
will mostly be centered around messenger RNA in one shape or form.

A basic property of gene expression is that it is carefully regulated. The
bacteria \textit{Escherichia coli} (\textit{E. coli}) contains more than 4000
protein coding genes in its genome, but at a given time only a subset of these
are expressed. Which genes are expressed at a given time depends on the
environmental condition and physiological state of the cell. Examples of
environmental conditions are nutrient availability, temperature, salt
concentrations, signals from other cells, and toxins. Examples of physiological
states are the different stages of the cell division cycle. Each of these input
signals and cellular states states result in unique or partially overlapping
patterns of gene expression responses honed over the course of evolution. In
this way, gene expression shapes all aspects of life, and understanding more of
gene expression is to dive deeper into the question about what life is.

In this thesis we will study several different stages of gene expression.
Specifically, we will study i) the initiation of transcription, ii) the
termination of transcription, and iii) the initiation of translation. The work
on transcription and translation initiation has been done on \textit{E. coli}
(bacterial) cells, while the work on transcription termination has been done
with human (eukaryotic) cells. It is estimated that these two cell types,
bacterial and eukaryotic, shared a last common ancestor over two billion years
\cite{vellai_origin_1999}; yet throughout that time, the key players in this
thesis, the RNA polymerase and the ribosome, have remained structurally and
functionally conserved between them.

In the following chapter we will review those stages of transcription and
translation that are relevant for the work at hand. The reviews will cover both
the biological and computational/theoretical background needed to read the
thesis as a whole.
