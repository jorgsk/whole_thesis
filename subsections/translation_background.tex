%\addbibresource{/home/jorgsk/phdproject/bibtex/jorgsk.bib}
Chapters X and Y are about gene expression systems in bacteria and how
these systems are regulated by RNA secondary structures in the ribosome binding
site (RBS), mRNA stability, and codon usage bias. Here, we review each of those
topcis. We begin with a brief account of translation initiation and then talk
about how this process is regulated by RNA secondary structures, and how this
again affects RNA stability. We then proceed to translation elongation and how
this process is affected by condon usage bias.

\subsubsection{Brief overview of translation}
Translation is the conversion of the genetic code on an mRNA into an amino acid
sequence. The molecule responsible for this conversion is the ribosome, a
macromolecular complex of protein and ribosomal RNA. The ribosome consists of
two subunits, S30 and S50, which bind at the 5\p region of an mRNA.
Together, they read the genetic code in the form of nucleotide triplets (eg.
AAG) called codons. Each codon is matched with an amino acid and the amino
acids are joined to each other in the ribosome to form the primary protein
sequence which will fold into a functional confirmation.

Since bacteria have no nucleus, the ribosome can bind the 5\p end of an mRNA as
it is being synthesized by the RNA polymerase. The coupling of translation with
transcription is called co-transcriptional translation. The first ribosome that
initiates translation on an mRNA follows the transcribing RNA polymerase
closely and even pushes it, causing both of them to follow the same speed
\cite{proshkin_cooperation_2010}. In this way, the ribosome can prevent RNAP
from backtracking, which can increase the speed of transcription by reducing
RNAP pausing. Fascinatingly, it was recently shown that by preventing RNAP
backtracking in this way, ribosomes reduce collisions between RNAP and DNA
polymerases during chromosome duplication, and that reducing these
collisions leads to fewer errors during genome duplication
\cite{dutta_linking_2011}. This study links the seemingly disparate topics of
transcription elongation and genome integrity and is an example how
interconnected the cell is.

\subsubsection{Translation initiation: binding of the 16S RNA to the
Shine-Dalgarno sequence} A central ribosomal RNA (rRNA) in translation
initiation is the 16S RNA which is part of the S30 subunit. It was shown by
Shine and Dalgarno in 1974 that the last nine bases of the 16S RNA in
\textit{E. coli} are ACCUCCUUA \cite{shine_3-terminal_1974}. They suggested
that this sequence could bind to a previously discovered conserved
complementary motif in the 5\p untranslated region of mRNA to initiate
translation \cite{shine_3-terminal_1974}. This has later been confirmed and
although it is the canonical way in which translation initiation occurs in
prokaryotes \cite{nakagawa_dynamic_2010}, translation initiation can in some
cases happen without a Shine-Dalgarno sequence on the mRNA
\cite{skorski_highly_2006, boni_non-canonical_2001}.  The complementary motif
on mRNA later came to be known as the Shine-Dalgarno sequence, and the
complementary bases on 16S RNA became known as the anti Shine-Dalgarno
sequence. The Shine-Dalgarno (SD) sequence is often given as GGAGGA although
the core motif can be reduced to GGAG; this sequence is conserved across
bacteria and archaea with very little variation \cite{nakagawa_dynamic_2010}.
The SD sequence is found in the mRNA 5\p region at a distance of 3 to 10 bases
from the start codon \cite{chen_determination_1994-1}.

Since the SD sequence directs the ribosome to bind mRNA, the area around the
Shine-Dalgaro site is called the ribosome binding site (RBS) and occupies
roughly 30 nucleotides. Mutations in the ribosome binding site can greatly
affect the rate of translation initiation \cite{shultzaberger_anatomy_2001}.

Two things are achieved when the 16S RNA binds the SD site during translation
initiation. The first is that he S30 subunit becomes physically anchored to the
transcript through the RNA-RNA bond between the SD and anti-SD sequence.  The
second is that the start codon (often AUG) becomes aligned to the peptidyl site
on S30 at which peptide bond formation occurs. Since the SD sequence aligns the
ribosome to the start codon, the SD sequence is to the ribosome much like the
promoter is to RNAP: both sequence regions direct these macromolecules to the
site of \textit{de novo} synthesis.  Following this analogy, in same the way
that transcription factors can block transcription initiation by binding to
promoter regions, other RNA than 16S can bind the SD region to hinder
translation initiation. Two types of SD-binding non-16S RNA exist. One type are
small RNAs that bind to the RBS to occlude ribosome biding
\cite{storz_controlling_2004}. The type other are RNA bases from the mRNA
itself which have base paired with the SD sequence in a secondary structure
through RNA folding.

\subsubsection{RNA folding} To understand on RNA folding affects in the
ribosome binding site affects translation initiation, it is necessary to
understand how RNA folding of mRNA occurs. The basis for RNA folding is that
RNA, as DNA, can basepair with itself; C pairs with G and A pairs with U. As
the nascent RNA emerges from the RNA exit channel of RNAP it is immediately
free to fold into an energetically favorable confirmation. The folding of the
mRNA as it emerges is called co-transcriptional folding. A key question in
co-transcriptional RNA folding has been how comparable the time scales for RNA
folding and RNA synthesis are \cite{de_smit_translational_2003-1}. Whether or
not RNA folds co-transcriptionally depends on the timescale for folding being
shorter than the time scale of transcript production. For bacteria, nucleotides
are incorporated at a rate of between 20 and 80 ms per nucleotide, while
formation and dissociation of semi-stable structures like helices occur on the
10 to 100 $\mu$s timescale \cite{isambert_jerky_2009}. In other words, for each
nucleotide produced, there is on average time for 1000 refolding events. It
should be noted that the time needed for the spontaneous refolding of an RNA
structure depends on the binding strength of that structure.

How the mRNA folds as it is being transcribed is called the folding pathway of
the mRNA. The cell has evolved mechanisms to either ensure or prevent certain
transient secondary structures along the folding pathway. The purpose of this
is to ensure that a desired final secondary (and in the case of e.g tRNA,
tertiary) state is reached \cite{pan_rna_2006-1}.

\subsubsection{RNA secondary structures in the ribosome binding site} It was
showed early on that translation rates were strongly affected by secondary
structures at the RBS \cite{hall_role_1982}. By changing the position and free
energy of secondary structures in the RBS, Smit and Duin found that protein
levels could be varied over a range of 500 \cite{de_smit_secondary_1990},
presumably reflecting varying rates in translation initiation. In particular, a
non-linear relationship between the folding energy of structures in the RBS and
the translation initiation rate was found. Below a certain free energy value,
the translation initiation rate did not change with respect to the strength of
the secondary structure; but above that limit, translation initiation decreased
exponentially as the secondary structures in the RBS got stronger
\cite{de_smit_secondary_1990}.

Not only the Shine-Dalgarno sequence but the entire RBS sequence, including
sequences downstream the start codon \cite{seo_quantitative_2009} must in
general be unstructured for translation initiation to occur. It is therefore not
surprising that lack of strong secondary structures is a hallmark of ribosome
binding sites in several species \cite{gu_universal_2010}. The folding energy
of the RBS can even be used to distinguish between active genes and
pseudogenes, presumably because pseudogenes are no longer under the selection
pressure to keep the RBS weakly structured \cite{keller_reduced_2012}.

It should be noted that weak structures do not imply the absence of structures.
It can in general be expected that the ribosome must deal with structured
binding sites. To account for this, it was initially suggested that ribosomes
would bind only as the structures in the RBS spontaneously unfolded
\cite{de_smit_translational_1994}. However, it was later appreciated that the
time scale for RNA folding and unfolding are orders of magnitude faster than
the time scale for ribosome binding to RNA. This led to the suggestion that
ribosomes could bind so called ribosome standby site close to the RBS where
they could slide into place once the secondary structure opened
\cite{de_smit_translational_2003-1}. This was later partially supported by a
study that showed that that the ribosome together with translation initiation
factors (IF) may unwind secondary structures at the RBS presumably from
ribosome standby sites \cite{studer_unfolding_2006}. However, the exact
mechanism for how the ribosome would unwind RNA structures is not clear.

\subsubsection{Modifying the RBS to increase gene expression}
It was early on realized that gene expression could be varied by mutagenizing
elements in the RBS \cite{warburton_increased_1983}. It has been found that
mutations upstream the SD sequence as well as in the first coding regions
affect translation initiation rates \cite{park_design_2007}
\cite{care_translation_2007}, in general due to modifying the RNA secondary
structure around the RBS. Making synonymous codon changes in the early codons
of a gene to reduce secondary structures rates has been suggested as a
convenient method to enhance the expression of foreign genes in bacteria
\cite{cebe_rapid_2006}.

Another approach than altering the already-present RBS of a gene is to
introduce a new RBS in the form of a 5\p fusion partner
\cite{lavallie_gene_1995}. The fusion partner is usually the 5\p end of an gene
(5\p UTR and early coding region) that is known to be well expressed in the
organism of interest or has other useful properties (such as the His-tag for
protein purification). The usage of such a tag will add an extra peptide
sequence to the N-terminus of the protein. This peptide may later be cleaved
off by specific proteases or left in place if its presence is tolerated
\cite{esposito_enhancement_2006}.

When seeking to optimize the expression of a given gene, labs in practise often
turn to commercial providers of gene optimization, such as DNA 2.0 or
GenScript. However, there are published alternatives. In addition to what has
been already mentioned about fusion tags and RBS mutagenesis, the most
comprehensive tool yet published for optimization of translation initiation
is the RBS calculator \cite{salis_automated_2009}. This software takes into
account the match of the 16S RNA to the SD sequence, the distance from the SD
sequence to the start codon, the folding energy of any RBS secondary
structures, the type of start codon, and the folding energy of the ribosome
standby site. It should be noted that of these variables, the one that explains
most of the variation in gene expression is the folding energy of the RBS
secondary structures, see \cite{salis_automated_2009} in the supplementary
materials.

\subsubsection{RNA secondary structures affect RNA stability} RNA stability
means how long it takes from an RNA is produced until it is degraded. The more
stable a transcript is, the more time ribosomes have to translate it to produce
protein, meaning that more stable transcripts lead to more protein. Also, RNA
degradation is the final way to stop protein production, making it a way for
the cell to swiftly control protein production.  For these reasons, RNA
stability is carefully regulated in the cell.

There are two well-known ways in which RNA secondary structures affect RNA
stability. The first is by hairpin structures at the 5\p and 3\p ends. The
hairpin at the 5\p end protects the triphosphate of the 5\p nucleotide against
attack from the protein RppH. RppH converts 5\p triphosphates to 5\p
monophosphates \cite{deana_bacterial_2008}, and RNAs with a 5\p monophosphate
are in turn targets for RNAse E, an endonuclease which can initiate RNA
degradation \cite{mackie_ribonuclease_1998}. In this way, a 5\p hairpin
indirectly protects the mRNA against endonucleolytic attack. The 3\p hairpin
protects directly against degradation from 3\p exonucleases, which need RNA
with an unstructured 3\p end for their activity \cite{rauhut_mrna_1999}.

The second way RNA structures affect RNA stability is through their effect on
ribosome binding. It is well established that actively translated mRNA are more
stable that untranslated mRNA. The idea was put forward that the narrow spacing
between between translating ribosomes on the transcript prevented
endonucleolytic cleavage \cite{deana_lost_2005}. However, it has been shown in
at least two cases that ribosome binding confers transcript stabilization on
its own in the absence of translation \cite{wagner_efficient_1994,
hambraeus_5_2002}.  This indicates that narrow spacing between translating
ribosomes is not necessary for increased stability. The precise mechanism for
how ribosome binding prevents degradation is not clear \cite{deana_lost_2005}.

\subsubsection{The role of codon bias in gene expression and cellular fitness}
During translation elongation the ribosome matches incoming amino acids on
transfer RNAs (tRNA) to codons on the mRNA. Since there are 61 codons that tRNA
can bind to ($4^3$ minus three stop codons) but only 20 amino acids, it is
inevitable that several codons are associated with the same amino acid; this
fact has been called the redundancy of the genetic code. This redundancy and
the observation that some iso-coding codons were preferred over others puzzled
early workers on the genetic code, and still today these observations remain
partially unanswered (see a great review on the topic by Plotkin and Kudla
\cite{plotkin_synonymous_2011}).

Codons that code for the same amino acid are called synonymous codons, and the
preference of one synonymous codon over another is called codon usage bias.
Codon usage bias is a universal phenomenon as it is found throughout the
kingdom of life \cite{sharp_codon_1988}. In many species, codon bias is
especially strong in highly expressed genes, which seems to indicate that these
codons are especially suited for high rates of translation. This has led to
the suggestion that some codons are more ``efficient'' than others during
translation and that genes with efficient codons are translated more
rapidly \cite{moriyama_gene_1998}. It was eventually shown that the efficient
codons are those that have the highest copy numbers of the corresponding tRNA
genes \cite{reis_solving_2004, elf_selective_2003}. This seems to explain that
efficient codons are translated more rapidly and perhaps more accurately
because the corresponding tRNAs are more abundant in the cell.

A commonly used measure for codon bias is the codon adaptation index (CAI). It
measures how similar a gene's codon content is compared to the codons in highly
expressed genes in the same organism \cite{sharp_codon_1987}. The CAI has been
shown to correlate positively with gene expression levels in several species
\cite{duret_expression_1999, jansen_revisiting_2003}.

A practical consequence of codon bias is that when expressing for example a
human gene in a bacterial host, it is often necessary to change the codons in
the gene to the preference of the bacteria (called codon optimization) to
achieve high expression levels \cite{gustafsson_codon_2004}, because the
preferred codons between humans and bacteria are different.

Two recent publications have shed light on the effect that codon bias has on
the rate of translation. In the first, Kudla et al. made random synonymous
codons changes in the green fluorescent protein (GFP) gene to generate a library
of 154 GFP genes with on average 114 different codons each
\cite{kudla_coding-sequence_2009}. They found no correlation between CAI and
the expression level of the GFP variants. On the other hand, CAI was highly
correlated with cell growth rates. Expression levels correlated only with the
RNA folding free energy around the start codon; but RNA folding energy did not
correlate with growth rates. The authors conclude from this that codon bias
exists in highly expressed genes not to optimize translation rates, but to
optimize overall cellular fitness. The hypothesis is that if highly expressed
genes contain codons for which there are few tRNA, ribosomes would pause
often on these genes, causing fewer ribosomes to be available to the rest of
the mRNA pool \cite{kudla_coding-sequence_2009}.

In the second paper, Tuller et al. \cite{tuller_evolutionarily_2010} examined
the codon bias of 27 organisms from all three domains of life using the tRNA
adaptation index (tAI). This measure ranks each codon with the copy number of
the associated tRNA in the genome \cite{tuller_evolutionarily_2010}. They found
a species-wide trend by which genes tend to have inefficient codons in the early
coding region (first 30-70 codons), but efficient codons in mid and late coding
regions. The early inefficient codons were labeled a slowly translating ramp.
Their hypothesis is that by reducing the speed during early translation,
ribosomes are more evenly spaced out during elongation, which would reduce
collisions between them and thereby increase the overall translation efficiency
in the cell.

A possible criticism for the Tuller et al. paper is that the
early codons of mRNA are also under the selective pressure to reduce mRNA
folding \cite{gu_universal_2010}. This would reduce the degrees of freedom for
selection of optimal codons in the early codon region, which could partly
explain the ramp effect \cite{plotkin_synonymous_2011}.

Finally, there are other sources of codon bias. One is the specific order that
codons appear in \cite{cannarozzi_role_2010}. Another is the avoidance of
message-bearing motifs like the Shine-Dalgarno element in the coding sequence.
It was shown in \textit{E. coli} that when Gly-Gly amino acid pairs are coded
for in a gene, the most common codon pair is GGC-GGC, which out of all possible
Gly codon pairs has the lowest possible affinity for the anti-Shine Dalgarno
sequence. On the other hand, the rarest codon pair for Gly-Gly was GGA-GGU,
which is an exact match to the Shine-Dalgarno sequence
\cite{li_anti-shine-dalgarno_2012}. In the same study it was shown that
Shine-Dalgarno sequences inside coding regions could cause rebinding of the
16SRNA during translation elongation and thereby cause translation pausing.
Curiously, this rebinding behavior is the same as found for the sigma factor
during transcription elongation by RNA polymerase \cite{mooney_sigma_2005}.
