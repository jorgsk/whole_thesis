%\addbibresource{/home/jorgsk/phdproject/bibtex/jorgsk.bib}
Chapters X and Y are about gene expression systems in bacteria and how these
systems are regulated by codon usage bias, mRNA stability, and RNA secondary
structures in the ribosome binding site (RBS). Here, we review each of these
topics. We begin with a brief account of translation initiation and then talk
about how this process is regulated by RNA secondary structures, and how this
again affects RNA stability. We then proceed to translation elongation and how
this process is affected by condon usage bias.

\subsubsection{Brief overview of translation}
Translation is the conversion of the genetic code of an mRNA into an amino acid
sequence. The molecule responsible for this conversion is the ribosome, a
macromolecular complex of RNA and protein. The ribosome consists of two
subunits, S30 and S50, which initiate translation by binding in the 5\p region
of an mRNA. Together, they read the genetic code in the form of nucleotide
triplets (eg. AAG) called codons. Each codon is matched with an anticodon in a
transfer RNA (tRNA) which carries the amino acid that corresponds to this
codon. The amino acids from the tRNAs are joined to each other in the ribosome
to form the primary protein sequence which eventually will fold into a
functional conformation.

Since bacteria have no nucleus, the ribosome can bind the 5\p-end of an mRNA at
the same time as the mRNA is being synthesized by the RNA polymerase. The
coupling of translation with transcription is called co-transcriptional
translation. The first ribosome that initiates translation on an mRNA follows
the transcribing RNA polymerase closely and even pushes it, causing both of
them to follow the same speed \cite{proshkin_cooperation_2010}. In this way,
the ribosome can prevent RNAP from backtracking, which can increase the speed
of transcription by reducing RNAP pausing. Fascinatingly, it was recently shown
that by preventing RNAP backtracking in this way, ribosomes reduce collisions
between RNAP and DNA polymerases during chromosome duplication, and that
reducing these collisions leads to fewer errors during genome duplication
\cite{dutta_linking_2011}. This study links the seemingly disparate topics of
translation and genome integrity and is an example how interconnected the cell
is.

\subsubsection{Translation initiation: binding of the 16S RNA to the
Shine-Dalgarno sequence} A central ribosomal RNA in translation initiation is
the 16S RNA which is part of the S30 subunit. It was shown by Shine and
Dalgarno in 1974 that the last nine bases of the 16S RNA in \textit{E. coli}
are ACCUCCUUA \cite{shine_3-terminal_1974}. They suggested that this sequence
could hybridize to a previously discovered conserved complementary motif close
to the start codon in the 5\p untranslated region of mRNA to initiate
translation \cite{shine_3-terminal_1974}. That this actually happens was later
confirmed and it is now known to be the canonical method of translation
initiation in prokaryotes (bacteria and archaea) \cite{nakagawa_dynamic_2010}.
Examples of translation initiation occurring without 16S RNA hybridization have
been found, but they are exceptions rather than the rule
\cite{skorski_highly_2006, boni_non-canonical_2001}.

The complementary motif on mRNA that 16S RNA hybridizes to came to be known as
the Shine-Dalgarno sequence, and the complementary bases on 16S RNA became
known as the anti Shine-Dalgarno sequence. The Shine-Dalgarno (SD) sequence is
located 3 to 10 bases downstream the start codon. Its sequence is often given
as GGAGGA although the core motif can be reduced to GGAG. This sequence is
conserved across prokaryotes with very little variation
\cite{nakagawa_dynamic_2010}. 

When the 16S RNA binds the SD site during translation initiation, the S30
subunit becomes physically anchored to the transcript through the RNA-RNA bond
between the SD and anti-SD sequence. As well, the start codon (mostly AUG in
\textit{E. coli}) becomes aligned to the peptidyl site on S30. The peptidyl
site of the ribosome is akin to the active site on RNAP in that the peptidyl
site is where amino acid bond formation occurs. Thus, the binding of the
ribosome to the SD sequence makes the ribosome ready to accept the initiating
tRNA.

After the ribosome has bound mRNA, it covers an area +/- 15 nt around the start
codon; this area is called the ribosome binding site (RBS)
\cite{kozak_regulation_2005}. By mutagenizing the RBS and the area around it it
is possible to greatly vary the rate of translation initiation. This is
especially true for the SD sequence, since its complementarity to 16S RNA
directly influences the affinity of the ribosome for the mRNA.

Since the SD sequence directs the ribosome to the start codon, the SD sequence
has the same function for the ribosome as the promoter has for RNAP. Both
sequences direct these macromolecules to the site of \textit{de novo}
synthesis. Following this analogy, in same the way that transcription factors
can block transcription initiation by binding to promoter regions, other RNA
than 16S can bind the SD region to block translation initiation. There are two
types of non-16S SD-binding RNA that are characterized. One type is small RNAs
called microRNA \cite{storz_controlling_2004}. The other type is actually the
mRNA which contains the SD sequence itself: during folding of the mRNA, a
sequence on the mRNA, e.g. CCUCCU, may base pair with its SD sequence, thereby
preventing access from the 16S RNA. RNA folding is a potent regulator of
transcription initiation \cite{hall_role_1982, de_smit_secondary_1990} and a
central topic in this thesis. RNA folding and the effect of RNA folding on
transcription initiation will now be covered in more detail.

\subsubsection{Co-transcriptional RNA folding}
To understand how the folding of RNA in the RBS affects translation initiation,
it is necessary to understand how the folding of mRNA occurs. The basis for RNA
folding is that RNA, as DNA, can basepair with itself; C pairs with G and A
pairs with U. The reason why RNA folds at all is that under physiological
conditions, base-paired nucleotides are more thermodynamically favorable than
free nucleotides \cite{onoa_rna_2004}. Folded RNA is often depicted as a series
of so called hairpins with base-paired stems and loops on top.

As the nascent RNA emerges from the RNA exit channel of RNAP it is immediately
free to fold into an energetically favorable conformation. The folding of the
mRNA as it emerges is called co-transcriptional folding. A key question in
co-transcriptional RNA folding has been how comparable the time scales for RNA
folding and RNA synthesis are \cite{de_smit_translational_2003-1}. Whether or
not RNA folds co-transcriptionally depends on the timescale for folding being
shorter than the time scale for RNA synthesis. In bacteria, RNA synthesis
happens at a rate of between 20 and 80 ms per nucleotide, while formation and
dissociation of semi-stable structures like helices occur on the 10 to 100
$\mu$s timescale \cite{isambert_jerky_2009}. In other words, for each
nucleotide produced, there is on average time for 1000 refolding events. It
should however be noted that the time needed for the spontaneous refolding of
an RNA structure depends on binding strength of that structure.

The folding pathway is the exact sequence of folding steps the mRNA undergoes
as it is synthesized. The cell has evolved mechanisms to either ensure or
prevent certain transient secondary structures along the folding pathway. The
purpose of this is to ensure that a desired final secondary (and in the case of
e.g tRNA, tertiary) structure is reached \cite{pan_rna_2006-1}.

\subsubsection{RNA secondary structures in the ribosome binding site}
It was shown early on that translation rates are strongly affected by secondary
structures at the RBS \cite{hall_role_1982}. In a seminal study, Smit and Duin
changed both the location and the binding strength of secondary structures in
the RBS of an mRNA, and found that protein levels could be varied over a range
of 500 \cite{de_smit_secondary_1990}, presumably reflecting varying rates in
translation initiation caused by these changes. In particular, they found a
non-linear relationship between the folding energy of structures in the RBS and
the translation initiation rate: below a certain threshold, the translation
initiation rate did not change with respect to the strength of the secondary
structure; but above that limit, translation initiation decreased exponentially
as the secondary structures in the RBS got stronger
\cite{de_smit_secondary_1990}.

Not only the SD sequence but the entire RBS sequence must in general be
unstructured for translation initiation to occur \cite{seo_quantitative_2009}.
It is therefore not surprising that the absence of strong secondary structures
is a hallmark of ribosome binding sites in several species
\cite{gu_universal_2010}. The folding energy of the RBS can even be used to
distinguish between active genes and pseudogenes in \textit{E.  coli}
(pseudogenes are genes which have lost their protein coding ability). This is
presumably because pseudogenes are no longer under selection pressure to avoid
strong secondary structures in the RBS \cite{keller_reduced_2012}.

It should be noted that although there are weak structures in the RBS it does
not imply the total absence of structures. It can in general be expected that
the ribosome must deal with structured binding sites. To account for this, it
was initially suggested that ribosomes would bind only as the structures in the
RBS spontaneously unfolded \cite{de_smit_translational_1994}. However, it was
later appreciated that the time scale for RNA folding and unfolding are orders
of magnitude faster than the time scale for ribosome binding to RNA. This led
to the suggestion that ribosomes could bind a so called ribosome standby site
close to the RBS from where they could approach the RBS when the secondary
structures there unfold \cite{de_smit_translational_2003-1}. This has however
not been conclusively demonstrated, but was partially supported by a study that
showed that that the ribosome, together with translation initiation factors,
may unwind secondary structures at the RBS presumably from ribosome standby
sites \cite{studer_unfolding_2006}.

\subsubsection{Modifying the RBS to increase gene expression}
It has been long known that mutagenizing elements in the RBS is way to increase
gene expression \cite{warburton_increased_1983}. The mutations in the RBS that
do not land in the SD sequence or the start codon affect translation
initiation primarily by modifying the RNA secondary structure around the RBS
\cite{park_design_2007, care_translation_2007}. Often these mutations are made
in the 5\p untranslated region (UTR) since base changes in this region do not
affect the final protein product. However, it is also common to make synonymous
codon changes in the first codons of a gene \cite{cebe_rapid_2006}, since this
approach neither affects the final protein product (synonymous mutations are
mutations to codons that do not change the amino acid sequence of the protein).

Another approach than mutagenizing the RBS sequence of a gene is to introduce a
new RBS in the form of a 5\p fusion partner \cite{lavallie_gene_1995}. The
fusion partner is usually the 5\p-end (5\p UTR and early coding region) of a
gene that is already known to be well expressed in the host organism or has
other useful properties (such as the His-tag for protein purification). When
using a 5\p fusion partner the peptide coded by the early coding region of the
fusion partner will be added to the N-terminus of the protein which is being
expressed. This peptide may later be cleaved off by specific proteases or left
in place if its presence is tolerated \cite{esposito_enhancement_2006}.

When labs want to optimize the expression of a given gene, they often turn to
commercial providers of gene optimization, such as DNA 2.0 or GenScript.
However, there are published alternatives. In addition to what has been already
mentioned about fusion tags and RBS mutagenesis, the most comprehensive tool
yet published for optimization of translation initiation is the RBS calculator
\cite{salis_automated_2009}. This software takes into account several sequence
dependent variables that have been shown to play a role in translation
initiation: the match of the 16S RNA to the SD sequence, the distance from the
SD sequence to the start codon, the folding energy of any RBS secondary
structures, the type of start codon, and the folding energy of the ribosome
standby site. The tool will suggest sequence changes to these regions that will
result in a calculated optimal expression for the gene of interest.

\subsubsection{RNA secondary structures affect RNA stability}
RNA secondary structures do not only affect ribosome binding, but they also
affect the stability of RNA. RNA stability is a term used to describe the
half-life of RNA in the cell; it is a measure of how long it takes from an RNA
is produced until it is degraded. If all else is equal, a stable transcript
would result in more protein being produced from it than an unstable
transcript. As well, after an mRNA is produced from a gene, RNA degradation is
the final way to stop protein expression from that gene. For these reasons, RNA
stability is carefully regulated in the cell.

There are two well-known ways in which RNA secondary structures affect RNA
stability. The first is by hairpin structures at the 5\p and 3\p ends of the
RNA. The direct effect of a hairpin at the 5\p-end of an RNA is that the
hairpin protects the 5\p-most nucleotide in the RNA against attack from the
protein RppH. RppH changes 5\p nucleotide by removing a phosphate group
\cite{deana_bacterial_2008}. If the 5\p nucleotide lacks this phosphate group,
the whole RNA becomes a target for targets for RNAse E, an endonuclease which
can initiate RNA degradation \cite{mackie_ribonuclease_1998}. In this way, a
5\p hairpin indirectly protects the mRNA against degradation by RNAse E. The
function of the 3\p hairpin is more direct. It protects directly against
degradation from RNAse which require RNA with an unstructured 3\p-end for their
activity \cite{rauhut_mrna_1999}.

The second way RNA structures affect RNA stability is through their already
mentioned effect on ribosome binding. It is well established that actively
translated mRNA are more stable that untranslated mRNA. This led to the idea
that the narrow spacing between translating ribosomes on the transcript would
prevented cleavage by RNAse \cite{deana_lost_2005}. However, it has been shown
in at least two cases that ribosome binding confers transcript stability on its
own in the absence of translation \cite{wagner_efficient_1994,
hambraeus_5_2002}. This indicates that narrow spacing between translating
ribosomes is not necessary for increased stability.  However, the precise
mechanism for how ribosome binding prevents degradation is not clear
\cite{deana_lost_2005}.

\subsubsection{The role of codon bias in gene expression and cellular fitness}
During translation elongation the ribosome matches incoming amino acids on
transfer RNAs (tRNA) to codons on the mRNA. There are 64 ($4^3$) codons in the
genomes of nearlt all organisms studied to date. Genreally, 61 of these match
the anti-codons on tRNA and the remaining three are stop codons. On the other
hand there are only 20 amino acids which should be matched to the 61 codons.
Therefore it is inevitable that several codons are associated with the same
amino acid; this fact has been called the redundancy of the genetic code. The
redundancy of the genetic code and the observation that some iso-coding codons
are preferred over others puzzled early workers on the genetic code. Still
today these observations remain partially unanswered (see a great review on the
topic by Plotkin and Kudla \cite{plotkin_synonymous_2011}).

Codons that code for the same amino acid are called synonymous codons, and the
preference of one synonymous codon over another is called codon usage bias.
Codon usage bias is a universal phenomenon as it is found throughout the
kingdom of life \cite{sharp_codon_1988}. In many species, codon bias is
especially strong in highly expressed genes, which seems to indicate that the
over-represented codons in these genes are especially suited for high rates of
translation.  This has led to the suggestion that some codons are more
``efficient'' than others during translation and that genes with efficient
codons are translated more rapidly \cite{moriyama_gene_1998}. It was eventually
shown that the efficient codons are those that have the highest copy numbers of
the corresponding tRNA genes \cite{reis_solving_2004, elf_selective_2003}. This
seems to explain that efficient codons are translated more rapidly because the
corresponding tRNAs are more abundant in the cell, shortening the waiting time
between each tRNA binds in the tRNA acceptor site of the ribosome.

A commonly used measure for codon bias is the codon adaptation index (CAI). It
measures how similar a gene's codon content is compared to the codons in highly
expressed genes in the same organism \cite{sharp_codon_1987}. The CAI has been
shown to correlate positively with gene expression levels in several species
\cite{duret_expression_1999, jansen_revisiting_2003}. The CAI has often been
used for codon optimization genes for heterologous expression. (Heterologous
expression is the expression of a gene when the gene and the host cell are from
different species.) For example, codon optimization is often performed to
achieve high expression levels when expressing human genes in bacterial hosts
\cite{gustafsson_codon_2004}. The reason codon optimization is necessary is
that the codon usage bias differs between species. A codon which is rapidly
translated in human may be slowly translated in \textit{E. coli}.

Two recent publications have shed light on the effect that codon bias has on
the rate of translation. In the first, Kudla et al. made random synonymous
codons changes in the green fluorescent protein (GFP) gene to generate a
library of 154 GFP gene variants with on average 114 different codons each
\cite{kudla_coding-sequence_2009}. The GFP variants displayed a 250 fold
variation in expression in \textit{E. coli}, and caused a marked difference in
cell culture growth rates. The CAI of the GFP variants did not correlate with
their expression levels, but instead correlated with the growth rate of the
cells. On the other hand, the strength of secondary structures around the start
codon correlated strongly with GFP expression, but did not show any correlation
with cell growth rates. The authors conclude from this that codon bias exists
in highly expressed genes not to optimize translation rates, but to optimize
overall cellular fitness. The hypothesis is that if highly expressed genes
contain codons for which there are few tRNA, ribosomes would pause often on
these genes, causing fewer ribosomes to be available to the rest of the mRNA
pool, slowing the cellular growth rate \cite{kudla_coding-sequence_2009}

In the second paper, Tuller et al. \cite{tuller_evolutionarily_2010} examined
the codon bias of 27 organisms from all three domains of life using the tRNA
adaptation index (tAI), as similar measure to the CAI. (Briefly, the tAI
measure ranks each codon with the copy number of the associated tRNA in the
genome \cite{tuller_evolutionarily_2010}). They found a species-wide trend
where genes tend to have inefficient codons in the early coding region (first
30-70 codons), but efficient codons in mid and late coding regions. The early
inefficient codons were labeled a slowly translating ``ramp''. Their hypothesis
is that by reducing the speed during early translation, ribosomes are more
evenly spaced out during elongation, which reduces collisions between them and
thereby increases the overall translation efficiency in the cell.

A criticism of the Tuller et al. paper is that the early codons of mRNA are
also under the selective pressure to reduce mRNA folding
\cite{gu_universal_2010}. This would reduce the degrees of freedom for
selection of optimal codons in the early codon region, which could partly
explain the ramp effect \cite{plotkin_synonymous_2011}.

Finally, there are other sources of codon bias than have been mentioned so far.
One is the specific order that codons appear in, which has been linked to more
efficient recycling of tRNA \cite{cannarozzi_role_2010}. Another is the
avoidance of message-bearing motifs like the Shine-Dalgarno element in the
coding sequence. It was shown in \textit{E. coli} that when Gly-Gly amino acid
pairs are coded for in a gene, the most common codon pair is GGC-GGC, which out
of all possible Gly codon pairs has the lowest possible affinity for the
anti-Shine Dalgarno sequence. On the other hand, the rarest codon pair for
Gly-Gly was GGA-GGU, which is an exact match to the Shine-Dalgarno sequence
\cite{li_anti-shine-dalgarno_2012}. In the same study it was shown that
Shine-Dalgarno sequences inside coding regions could cause rebinding of the
16S RNA during translation elongation and thereby cause translation pausing.
Interestingly, this rebinding behavior is the same as found for the sigma
factor during transcription elongation by RNA polymerase
\cite{mooney_sigma_2005}, as was previously mentioned. This shows that there is
a price to be had for the strong binding affinities of sigma with the promoter
and 16S RNA for the Shine-Dalgarno sequence. It remains to be seen if a similar
avoidance of promoter-like sequences exist in the early coding regions of
bacterial genes.
