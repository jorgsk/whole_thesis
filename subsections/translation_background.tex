%\addbibresource{/home/jorgsk/phdproject/bibtex/jorgsk.bib}
Chapters X and Y deal with how gene expression systems in bacteria and how
these systems are regulated by RNA secondary structures in the ribosome binding
site (RBS), mRNA stability, and codon usage bias. Here, we review each of those
topcis. We begin with a brief account of translation initiation and then talk
about how this process is regulated by RNA secondary structures, and how this
again affects RNA stability. We then proceed to translation elongation and how
this process is affected by condon usage bias.

\subsubsection{Brief overview of translation}
Translation is the conversion of the genetic code on an mRNA into an amino acid
sequence. The molecule responsible for this conversion is the ribosome, a
macromolecular complex of protein and ribosomal RNA. The ribosome consist of
two subunits, S30 and S50, which bind sequentially at the 5' region of an mRNA.
Together, they read the genetic code in the form of nucleotide triplets called
codons. Each codon is matched with an amino acid and the amino acids are joined
to each other to form the primary protein.

Since bacteria have no nucleus, the ribosome can bind the 5' end mRNA as it is
being synthesized by the RNA polymerase, leading to co-transcriptional
translation. The first ribosome that initiates
translation on an mRNA follows the transcribing RNA polymerase closely and even
pushes it, causing both of them to follow the same speed 
\cite{proshkin_cooperation_2010}. In this way, the ribosome can prevent RNAP
from backtracking, which can increase the speed of transcription by reducing
RNAP pausing. Fascinatingly, it was recently shown that by preventing RNAP
backtracking, the ribosomes reduce collisions between RNAP and DNA polymerases
performing chromosome duplication, and that reducing these collisions leads to
fewer errors during genome duplication \cite{dutta_linking_2011}.

\subsubsection{Translation initiation: binding of the 16S RNA to the Shine-Dalgarno sequence}
A central ribosomal RNA (rRNA) in translation initiation is the 16S RNA which
is part of the S30 subunit. It was shown by Shine and Dalgarno in 1974 that the
last nine bases of the 16S RNA in \textit{E. coli} are ACCUCCUUA
\cite{shine_3-terminal_1974}. They suggested that this sequence could bind to a
previously discovered conserved complementary motif in messenger RNAs 5' UTR
regions to initiate translation \cite{shine_3-terminal_1974}. This has later
been confirmed and is now considered the canonical way in which translation
initiation occurs in prokaryotes \cite{nakagawa_dynamic_2010}, although there
are rare instances of SD-indpendent translation initiation
\cite{skorski_highly_2006, boni_non-canonical_2001}. The
complementary motif on mRNA later came to be known as the Shine-Dalgarno
sequence, and the complementary bases on 16S RNA became known as the anti
Shine-Dalgarno sequence. The Shine-Dalgarno (SD) sequence is often given as
GGAGGA although the core motif can be reduced to GGAG; this sequence is
conserved across bacteria and archea with very little variation
\cite{nakagawa_dynamic_2010}. The SD sequence is found usually in the mRNA 5'
region at a distance of 3 to 10 bases from the start codon
\cite{chen_determination_1994-1}. The area around the Shine-Dalgaro site is
called the ribosome binding site (RBS) and occupies roughly 30 nucleotides.
This whole area is vital for translation initiation and by mutating the area
the rate of translation initiation can be varied
\cite{shultzaberger_anatomy_2001}.

When the 16S RNA binds the SD site during translation initiation, two things
are achieved. The first is that he S30 subunit becomes physically anchored to
the transcript through the RNA-RNA bond between the SD and the anti-SD
sequence. The second is that the start codon (often AUG) becomes aligned to the
peptidyl site on S30 at which peptide bond formation occurs. Since the SD
sequence aligns the ribosome to the start codon, the SD sequence is to the
ribosome much like the promoter is to RNAP: both sequence regions direct these
macromolecules to the site of \textit{de novo} synthesis. Following this
analogy, in same the way that transcription factors can block transcription
initiation by binding to promoter regions, other RNA than 16S can bind the SD
region to hinder translation initiation. This mechanism makes it possible for
the cell to control translation initiation. Two types of SD-binding non-16S RNA
exist. One type are small RNAs that bind to the RBS to occlude ribosome biding
\cite{storz_controlling_2004}. The type other are RNA bases from the mRNA
itself which have base paired with the SD sequence in a secondary structure
through RNA folding.

%XXX start here with more review
\subsubsection{RNA folding}
To understand on RNA folding affects the SD-site, it is necessary to understand
how RNA folding happens. The basis for RNA folding is that RNA, as DNA, can
basepair with itself; C pairs with G and A pairs with U. As the nascent RNA
emerges from the RNA exit channel of RNAP it is immediately free to fold into
an energetically favorable confirmation. The folding of the mRNA as it emerges
is called co-transcriptional folding. A key question in co-transcriptional RNA
folding has been how comparable the time scales for RNA folding and RNA
synthesis are \cite{de_smit_translational_2003-1}. Whether or not RNA folds
co-transcriptionally depends on the timescale for folding being shorter than
that of transcript production. For bacteria, nucleotides are incorporated at a
rate of between 20 and 80 ms per nucleotide, while formation and dissociation
of semi-stable helices occur on the 10 to 100 $\mu$s timescale
\cite{isambert_jerky_2009}. In other words, for each nucleotide produced, there
is on average time for 1000 refolding events. It should be noted, the time
needed for the spontaneous refolding of an RNA structure depends on the binding
strength of that structure. The cell has evolved mechanisms to either ensure or
prevent certain transient secondary structures during the folding pathway. The
purpose of this is to ensure that a desired final secondary (and in the case of
some RNA, tertiary) state is reached \cite{pan_rna_2006-1}.

\subsubsection{RNA secondary structures in the ribosome binding site}
It was early on showed that translation rates were stronly affected by
secondary structures at the RBS; changes to the secondary structures were found
to affect translation rates by a factor of over 500
\cite{de_smit_secondary_1990}. In particular, a non-linear relationship between
the folding energy of structures in the RBS and the translation initiation
rate was found: below a certain limit, the translation initiation rate is almost
invariant with respect to the strength of the secondary structure; above that
limit, the relationship is exponential:
\begin{equation*}
	\text{Translation rate} \approx exp^{\Delta G_{\text{RBS}}}
\end{equation*}

Not only the Shine-Dalgarno sequence but the entire RBS sequence, including
sequences downstream the start codon \cite{seo_quantitative_2009} must in
general be unstructured for translation initiaton to occur. It is therefore not
surprising that lack of strong secondary structures is a hallmark of ribosome
binding sites in \textit{E coli} \cite{gu_universal_2010} genome wide. The
folding energy of the RBS can even be used to distinguish between active genes
and pseudogenes \cite{keller_reduced_2012}, presumably because pseudogenes are
no longer under the selection pressure to keep the RBS weakly structured.

It should be noted that weak structures do not imply the absence of structures.
In general, it can be expected that the ribosome must deal with structured
binding sites. To account for this, it was initially suggested that ribosomes
would bind only as the structues in the RBS spontaneously unfolded
\cite{de_smit_translational_1994}. However, it was later apprecated that the
time scale for RNA folding and unfolding are orders of magnitude faster than
the time scale for free ribosome binding. This led to the suggestion that
ribosomes could bind so called ribosome standby site close to the RBS where
they could slide into place once the secondary structure opened
\cite{de_smit_translational_2003-1}. This was subsequently supported by a study
that showed that that the ribosome together with translation initiation factors
(IF) can unwind secondary structures at the RBS presumably from ribosome
standby sites \cite{studer_unfolding_2006}. The exact mechanism or the actor of
RNA unfolding is however not clear.

\subsubsection{Modifying RBS secondary structures to increase gene expression}
Given that secondary structures in the RBS have such a marked impact on
translation rates, many studies have varied the RBS-sequence for the purpose of
tuning gene expression \cite{cebe_rapid_2006} \cite{park_design_2007}
\cite{berg_expression_2009}. For example, it is often necessary to modify the
RBS sequence of eukaryotic genes when trying to express them in a prokaryotic
host \cite{care_translation_2007}.

In practise when seeking to optimize the expression of a given gene, labs turn
to commercial providers of gene optimization, such as DNA 2.0 or GenScript.
However, there are published alternatives. The most prominent of these is the
RBS calculator \cite{salis_automated_2009}. This software takes into account
the match of the 16S RNA to the SD sequence, the distance from the SD sequence
to the start codon, the folding energy of any RBS structures, the start codon,
and the folding energy of the ribosome standby site. Of these variables, the
one that correlatest most strongly with gene expression is the folding energy
of the RBS structures, see \cite{salis_automated_2009} in the supplementary
materials. By considering these variables, the software will suggest
modifications to a given gene sequence to set gene expression to a specified
level.

\subsubsection{RNA secondary structures affect RNA stability}
RNA stability means how long it takes from an RNA is produced until it is
degraded. The more stable a transcript is, the more time ribosomes have to
translate it to produce protein, meaning that more stable transcripts lead to
more protein. Also, RNA degradation is the final way to stop protein
production, making it a way for the cell to swiftly control protein production.
For these reasons, RNA stability is carefully regulated in the cell.

There are two well-known ways in which RNA secondary structures affect RNA
stability. The first is by hairpin structures at the 5' and 3' ends. The
hairpin at the 5' end protects the triphosphate of the 5' nucleotide against
attack from the protein RppH which converts 5' triphosphates to 5'
monophosphates \cite{deana_bacterial_2008}. In turn, RNAs with a 5'
monophosphate are targets for RNAse E, an endonuclease which can initiate RNA
degradation \cite{mackie_ribonuclease_1998}. In this way, a 5' hairpin
indirectly protects the mRNA against endonucleolytic attack. On the other hand,
the 3' hairpin protects directly against degradation from 3' exonucleases,
which need RNA with an unstructured 3' end for their activity
\cite{rauhut_mrna_1999}.

The second way RNA structures affect RNA stability is more indirect. As
mentioned, secondary structures at the RBS affects the rate of translation
initiation. The rate of translation again affects the spacing between
translating ribosomes on the transcript. Since it was early observed that
translated RNA is protected from degradation, the idea was put forth that
reduced spacing between translating ribosomes protects against endonucleolytic
cleavage by steric blocking. It has however been shown in at least two cases
that ribosome binding, but not translation, confer transcript stability
\cite{wagner_efficient_1994} \cite{hambraeus_5_2002}, indicating that narrow
spacing between translating ribosomes is not necessary for increased stability.
Still, the precise mechanism for how ribosome binding prevents degradation is
not clear \cite{deana_lost_2005}.

\subsubsection{Codon bias during translation elongation}
During translation elongation the ribosome matches incoming amino acids on
transfer RNAs (tRNA) to codons (nucleotide triplets; eg, ATG) on the mRNA. Each
codon corresponds to an amino acid, and in this way the RNA sequence is being
translated into a protein sequence. Since there are 61 codons that bind tRNA
($4^3$ minus three stop codons) but only 20 amino acids, it is inevitable that
several codons code for the same amino acid; this fact has been called the
redundancy of the genetic code. This redundancy and the fact that some
iso-coding codons were preferred to others puzzled early workers on the
genetic code, and still today these observations remain partially unanswered
(see a great review on the topic by Plotkin and Kudla \cite{plotkin_synonymous_2011}).

Codons that code for the same amino acid are called synonoymous codons, and the
preference of one synonymous codon over another is called codon usage bias.
Codon usage bias is found throughout the kingdom of life
\cite{sharp_codon_1988}, and it has been found in several species that codon
bias is especially strong in highly expressed genes, leading to the idea that
some tRNA are more ``efficient'' than others and that their usage leads to more
efficient translation \cite{moriyama_gene_1998}. Indeed, when expressing for
example a human gene in a bacterial host, it is often necessary to change the
codons in the gene to the preference of the bacteria (called codon
optimization) to achieve high expression levels \cite{gustafsson_codon_2004},
because the preferred codons between humans and bacteria are different.

Two recent publications have shed light on the role of codon bias in the rate
of translation. In the first, Kudla et al. selected synomymous codons at random
in the green fluorecent protein (GFP) gene to generate a library of 141 GFP
genes with on average 100 different codons each
\cite{kudla_coding-sequence_2009}.


Kudla + Ramp.
