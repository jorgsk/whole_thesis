%\addbibresource{/home/jorgsk/phdproject/bibtex/jorgsk.bib}
\subsubsection{Transcriptome sequencing with RNA-seq}
The work presented in this thesis and most other high-throughput studies of
polyadenylation rely on sequencing data made with Illumina sequencing
technologies. Here we briefly mention the stages of an RNA-seq
experiment that are relevant for interpreting subsequent results.

Briefly, RNA-seq is a technique to obtain a quantitative measure of the amount
of RNA in a sample. Typically, this sample will be all extracted RNA from a
cell culture. RNA-seq is used to compare the differential expression of
genes, to discover new genes, and to discover novel isoforms of known genes.
The output from an RNA-seq experiment are RNA fragments in the length of
50 to 500 nucleotides. These fragments, called reads, can be mapped back to the
genome, thereby showing where the RNA originated from. If a genome is not
available, it is possible to use the reads to assemble the transcriptome
\textit{de novo}.

The sequencing machines do not work directly on RNA. Therefore, RNA-seq relies
on the conversion of RNA to cDNA by a reverse transcriptase. To obtain good
results from a sequencing experiment, fragmented DNA in large concentrations is
required. To obtain this concentration, the RNA is convered to cDNA and
amplified with PCR. From the point of view of studying polyadenylation,
it is necessary that the poly(A) tail is preserved both during cDNA conversion
and subsequent PCR amplification. Several custom techniques have been developed
for the purpose of capturing the polyadenylation site with RNA-seq
\cite{ozsolak_comprehensive_2010, derti_quantitative_2012}, but a minimum
requirement is that poly(T) primers are used during cDNA conversion and PCR
amplificaiton.

After cDNA preparation and several rounds of PCR amplification, most poly(A)
tails will be reduced in length from the full 250 nt. This is because the
poly(T) primers will bind randomly in the poly(A) tail, something which will
gradually reduce the average length of the poly(A) tails.

It is common to sequence both ends of an RNA fragment. This is called paired
end sequencing. Paired end sequencing gives extra information and is
particularly useful when mapping reads over exon-intron junctions. For example,
if two reads from the same RNA fragment map over 5000 nucleotides apart, it is
likely that there was an exon-intron junction on the RNA fragment from which
they were sequenced.

\subsubsection{Sequencing errors and biases}
There are several errors and biases that creep into sequencing that must be
taken into account during the computational analysis of the output. Here we
discuss one error and one bias that is important for analysing Illumina short
paired end reads.

An error that is particularly important for the study of polyadenylation is
that sequencing performance degrades over single-nucleotide stretches
\cite{minoche_evaluation_2011}. This implies that when sequencing the poly(A)
sequence, other nucleotides than just A will be reported with a relatively high
probabilty. This means that when looking for poly(A) stretches in the output as
signatures of poly(A) sites, some variation from the homopolymer sequence must
be expected.

An important bias is that most sequences with poly(A) tails will be output in
the reverse-transcribed version. That is, 5\p CCCGAAAA 3\p will be output as 5\
TTTTCGGG 3\p. This is due to the RNA fragmenting and that sequencing
(performed by a DNA polyermase) happens in the 5\p to 3\p direction and that
each RNA is present in both its original and reverse transcribed form. During
sequencing, 50 to 150 nucleotides will be sequenced from each side of the around
300 nt fragment. If the poly(A) tail is the 50 last nucleotides of the 300 nt
fragment RNA, sequencing 150 nucleotides will never reach nucleotide 250 where
the poly(A) site is. However, the reverse transcribed version contains 50 Ts at
the 5\p end, which ensures that sequencing this version of the RNA gets the
whole poly(T) tail as well as 100 bases which can be mapped to the genome.
Even though the actual output poly(A) tail is in the reverse transcribed
poly(T) format, we will generally refer to both of them as a poly(A) tail.
