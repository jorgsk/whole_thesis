%\addbibresource{/home/jorgsk/phdproject/bibtex/jorgsk.bib}
\subsubsection{Summary of the paper}
This section refers to the paper by Kucharova et al. called ``5\p fusion
partners for recombinant expression in \textit{E. coli}'' which is attached in
the appendix.

The primary aim of the publication by Kucharova et al. was to express in
\textit{E. coli} a variant of the human interferon gene interferon-alpha called
\textit{inf-$\alpha$2b}. The protein product INF-$\alpha$2b is of
pharmaceutical interest as a dug for treating hepatitis C. The expression of a
foreign gene in a host organism is called heterologous expression, and carries
with it many challenges \cite{gustafsson_codon_2004}. Since heterologous
expression of INF-$\alpha$2b is necessary to produce the drug in large
quantities, it is important to investigate the limitations and mechanisms of
\textit{inf-$\alpha$2b} expression in \textit{E. coli}.

Kucharova et al reports that the \textit{inf-$\alpha$2b} gene is not
expressed in \textit{E.  coli} in its native form. A codon optimized version of
\textit{inf-$\alpha$2b} was obtained to investigate if the non-native codon
usage in the gene was behind the lack of expression. That is, the native human
codons were replaced with codons that are in high usage in \textit{E. coli}.
However, in spite of of codon optimization the gene was not expressed. In order
to detect the gene product with western blot, a 5\p fusion tag needed to be
added to the \textit{inf-$\alpha$2b} gene. This indicated that events occurring
in the 5\p region of the gene is involved in regulating the switch between
expression and nonexpresssion of the \textit{inf-$\alpha$2b} gene.

In their paper, Kucharova et al. suggest that translation initiation of the
\textit{inf-$\alpha$2b} transcript is a possible reason why the gene is not
expressed. That prompts them to reduce RNA secondary structures around the
ribosome binding site. This modification causes transcript levels to be greatly
increased. However, the increase in transcript level is not followed by an
increase in protein level. This suggests that further barriers than ribosme
binding lie behind the poor expression of \textit{inf-$\alpha$2b}. Having
established this, Kucharova et al. move on to investigate the effect of 5\p
terminal fusion peptides. Fusion peptides typically originate from the 5\p end
of a natively well-expressed gene and facilitate translation initiation of
heterologous mRNA. Kucharova et al. demonstrate that several short versions
of the celB fusion peptide increases expression of the \textit{inf-$\alpha$2b}
gene and several other genes. Finally, they improve the expression level with
the celB leader by screening a random mutagenesis library of celB around the
ribosome binding site. In conclusion, these fusion peptides probably facilitate
translation initiation both by having favorable secondary structure and also by
some downstream effect that could not be achieved with point mutations of the
\textit{inf-$\alpha$2b} gene alone.

\subsubsection{My contribution}
My contribution to the paper was to perform the bioinformatic analysis, partake
in the planning of some of the experiments, and to assist in writing.

Initially, I performed RNA secondary structure analysis for a set of
celB-\textit{inf-$\alpha$2b} variants with variable expression levels (data not
shown in the paper). This analysis could not conclude that secondary structures
were the reason behind the variation in expression; specifically it could not
conclude that strong secondary structures were behind the lack of expression of
the unmodified \textit{inf-$\alpha$2b} gene.

Nevertheless, we proceeded to try to optimize the secondary structures around
the ribosome binding site (RBS) of the original \textit{inf-$\alpha$2b} gene
without celB, because this is an approach that has worked previously for
heterologous expression \cite{de_smit_secondary_1990}. For this work, I made a
combinatorial library of synonymous codons for the first nine codons in the
\textit{inf-$\alpha$2b} coding region. I labeled each of the resulting set of
over 8000 synonymous variants with an index for the rarity of their codons and
the folding energy of the secondary structure around the RBS. Some selected
variants from this library were chosen based on their folding energy and rarity
of codon usage. When they were tested experimentally it was found that several
of them increased transcript levels but not protein levels. In addition, I
calculated the folding energy and codon bias of all sequence variants used in
the study, as found in the tables.
