%\addbibresource{/home/jorgsk/phdproject/bibtex/jorgsk.bib}

\subsubsection{Conceptual model for RNAP movement on DNA}

TODO include a figure of RNAP in all its pride and glory.

The movement of RNAP on DNA is thought to occur by a so called Brownian ratchet
mechanism. In this model, the backward movement of RNAP is structurally
prevented by the RNAP F bridge \cite{bar-nahum_ratchet_2005}, while the forward
motion is free to happen by Brownian motion. Still, pausing and backtracking
does happen during elongation, and much effort has been devoted to understand
exactly how RNAP moved on DNA. To analyze RNAP movement, the current model
assumes that the movement of RNAP is dependent only on the free energy
difference between the current state and all other possible states, such as a
paused state, a backtracked state, a terminating state, and a forward moving
state \cite{greive_thinking_2005}. This assumption relies on RNAP being in
chemical equilibrium between each step; and while this may not always hold true,
it does facilitate the formulation of the movement of RNAP as an equation \cite{greive_thinking}
\begin{equation}
	\Delta G_{\text{Total}} = \Delta G_{\text{RNA-DNA}} + \Delta
	G_{\text{DNA-DNA}} + \Delta G_{\text{RNAP}},
	\label{eq:rnap_energy_balance}
\end{equation}
This equation contains the terms currently believed to play a role in RNAP
movement, based on the interaction between RNAP and DNA. $\Delta
G_{\text{RNA-DNA}}$ is the $\sim$ 8-9 bp RNA-DNA helix inside RNAP, $\Delta
G_{\text{DNA-DNA}}$ is the $\sim$ 14-15 bp DNA-DNA bubble that RNAP keeps open
around it at all times, and $\Delta G_{\text{RNAP}}$ is the non-specific
energy state of the RNAP at a given position on DNA
\cite{greive_thinking_2005}. By calculating these energies for all possible
reaction pathways for RNAP, one can find the pathway that is most energetically
favorable.

This conceptual model seems orderly, but one should consider whether
historically the RNA-DNA and DNA-DNA variables have been included in the model
because they fit experimental data or because they simply have been possible to
measure compared to the other interactions involved in transcription elongation.

\subsubsection{Computational models of transcription elongation}
Since the first two terms of equation \eqref{eq:rnap_energy_balance} can be
calculated from published energy tables \cite{wu_temperature_2002,
santalucia_thermodynamics_2004}, several kinetic and thermodynamic models of
transcription elongation have been published
\cite{tadigotla_thermodynamic_2006-1, bai_sequence-dependent_2004,
guajardo_model_1997}. What they have in common is that in some form they
incorporate the terms from \eqref{eq:rnap_energy_balance}. In the case of
Tadigotla et al. \cite{tadigotla_thermodynamic_2006-1} the free energy from the
nascent RNA secondary structure close to the RNA exit channel is used as well;
this is for the purpose of modeling the effect of strong RNA hairpins in
preventing backtracking of paused RNAP.

These models have had partial predictive power, but more work needs to be done
before a truly descriptive model of transcription is at hand. The state of the
current models makes it clear that there is more to RNAP processivity than the
thermodynamic variables in equation \eqref{eq:rnap_energy_balance} on which
these models are based. Indeed, it was recently published that the 3\p
dinucleotide of the nascent RNA has a strong effect on translocation rates
\cite{hein_rna_2011}, showing that there are other factors at play than have
been previously suspected.

\subsubsection{A computational model of transcription initiation}
The key differences between transcription initiation and transcription
elongation are that during initiation i) RNAP is bound to the promoter via
sigma, ii) the DNA-DNA bubble is scrunched and pulled into RNAP, and iii) the
RNA-DNA hybrid lacks its full length until the active site has reached +8/9.
The consequence of these difference is that RNAP has a reaction pathway during
initiation that does not exist during elongation: abortive initiation.

Based on equation \eqref{eq:rnap_energy_balance} and the above differences, a
model for transcription initiation was previously published
\cite{xue_kinetic_2008}. This model attempts to cover explicitly both the
abortive process and the actual promoter escape event. Although the authors
manage to predict the maximum size of abortive transcript and the abortive to
productive ratio (APR) of the N25 promoter, they do not explain well the
abortive probabilities. Some criticism of this study is that only the N25,
N25anti and T7A1 promoters were investigated, even though a larger dataset was
available \cite{hsu_initial_2006}. Further, the T7A1 promoter varies not only
in the ITS sequence compared to the other two, but also in the core promoter
sequence. Variation in the core promoter sequence has been shown to affect
rates of abortive initiation \cite{vo_vitro_2003-1}, yet the authors have not
taken this variation into account.

In conclusion, this study gave the first quantitative model for transcription
initiation, and manages to reproduce some of the experimentally measured
parameters. However, the model is not used predictively, making it difficult to
measure how much of transcription initiation it truly captures. In the field of
transcription initiation, it remains to be published a quantitative model with
predictive power that gives new insight into the process itself.
