%\addbibresource{/home/jorgsk/phdproject/bibtex/jorgsk.bib}

\subsubsection{Conceptual model for RNAP movement on DNA}
As mentioned, RNAP moves on DNA during transcription by the cycle of
translocation and nucleotide incorporation. The forward -- as opposed to
backward -- motion of RNAP is thought to occur by a so called Brownian ratchet
mechanism. In this model, once a nucleotide has been incorporated, the backward
movement of RNAP (backtracking) is structurally prevented by a feature in RNAP
called the F bridge \cite{bar-nahum_ratchet_2005}. The forward motion, the
translocation step, is on the other hand free to happen by Brownian motion.

Nevertheless, pausing and backtracking do happen during elongation. To analyze
RNAP movement, the current conceptual model assumes that RNAP is in chemical
equilibrium between each nucleotide incorporation step
\cite{greive_thinking_2005}. This allows the reduction of RNAP movement into
the difference in free energy between the current state and all other
possible states, such as a paused state, a backtracked state, or a forward
moving state \cite{greive_thinking_2005}. While it is difficult to conclusively
demonstrate that the equilibrium assumption always holds true, it does
facilitate the formulation of the movement of RNAP as an equation:
\begin{equation}
	\Delta G_{\text{Total}} = \Delta G_{\text{RNA-DNA}} + \Delta
	G_{\text{DNA-DNA}} + \Delta G_{\text{RNAP}},
	\label{eq:rnap_energy_balance}
\end{equation}
This equation contains the terms currently believed to play a role in RNAP
movement, based on the interaction between RNAP and DNA. $\Delta
G_{\text{RNA-DNA}}$ is the free energy of the $\sim$ 8-9 bp RNA-DNA helix
inside RNAP, $\Delta G_{\text{DNA-DNA}}$ is the free energy of the $\sim$ 15 bp
DNA bubble that RNAP keeps open around it at all times, and $\Delta
G_{\text{RNAP}}$ is the non-specific free energy of other interactions between
RNAP, DNA, and RNA \cite{greive_thinking_2005}. By calculating these energies
for all possible reaction pathways for RNAP, one can presumably find the
pathway that is most energetically favorable, and thus map out the movement of
RNAP on DNA.

This conceptual model of RNAP movement seems orderly: one needs only to
calculate the change in free energy -- from available data -- to find out if
RNAP will move forward, backtrack, pause, or terminate at any given location on
DNA. But one should consider whether historically the RNA-DNA and DNA-DNA
variables have been included in the model because they fit experimental data or
because their measured values have simply been available in the literature for
decades.

\subsubsection{Computational models of transcription elongation}
% start here kthnx
Since the first two terms of equation \eqref{eq:rnap_energy_balance} can be
calculated from published energy tables \cite{wu_temperature_2002,
santalucia_thermodynamics_2004}, several kinetic and thermodynamic models of
transcription elongation have been published
\cite{tadigotla_thermodynamic_2006-1, bai_sequence-dependent_2004,
guajardo_model_1997}. What they have in common is that in some form they
incorporate the terms from \eqref{eq:rnap_energy_balance}. In the case of
Tadigotla et al. \cite{tadigotla_thermodynamic_2006-1} the free energy from the
nascent RNA secondary structure close to the RNA exit channel is used as well;
this is for the purpose of modeling the effect of strong RNA hairpins in
preventing backtracking of paused RNAP.

These models have had partial predictive power, but more work needs to be done
before a truly descriptive model of transcription is at hand. The state of the
current models makes it clear that there is more to RNAP processivity than the
thermodynamic variables in equation \eqref{eq:rnap_energy_balance} on which
these models are based. Indeed, it was recently published that the 3\p
dinucleotide of the nascent RNA has a strong effect on translocation rates
\cite{hein_rna_2011}, showing that there are other factors at play than have
been previously suspected.

\subsubsection{A computational model of transcription initiation}
The key differences between transcription initiation and transcription
elongation are that during initiation i) RNAP is bound to the promoter via
sigma, ii) the DNA-DNA bubble is scrunched and pulled into RNAP, and iii) the
RNA-DNA hybrid lacks its full length until the active site has reached +8/9.
The consequence of these difference is that RNAP has a reaction pathway during
initiation that does not exist during elongation: abortive initiation.

Based on equation \eqref{eq:rnap_energy_balance} and the above differences, a
model for transcription initiation was previously published
\cite{xue_kinetic_2008}. This model attempts to cover explicitly both the
abortive process and the actual promoter escape event. Although the authors
manage to predict the maximum size of abortive transcript and the abortive to
productive ratio (APR) of the N25 promoter, they do not explain well the
abortive probabilities. Some criticism of this study is that only the N25,
N25anti and T7A1 promoters were investigated, even though a larger dataset was
available \cite{hsu_initial_2006}. Further, the T7A1 promoter varies not only
in the ITS sequence compared to the other two, but also in the core promoter
sequence. Variation in the core promoter sequence has been shown to affect
rates of abortive initiation \cite{vo_vitro_2003-1}, yet the authors have not
taken this variation into account.

In conclusion, this study gave the first quantitative model for transcription
initiation, and manages to reproduce some of the experimentally measured
parameters. However, the model is not used predictively, making it difficult to
measure how much of transcription initiation it truly captures. In the field of
transcription initiation, it remains to be published a quantitative model with
predictive power that gives new insight into the process itself.
