%\addbibresource{/home/jorgsk/phdproject/bibtex/jorgsk.bib}

\subsubsection{Conceptual model for RNAP movement on DNA}
As mentioned in the previous section, RNAP moves on DNA during transcription by
a cycle of translocation and nucleotide incorporation. The fact that RNAP tends
to move forward and not backward is thought to be the result of a so called
Brownian ratchet mechanism. In this model, once a nucleotide has been
incorporated at the active site, the backward movement of RNAP (backtracking)
is structurally prevented by a feature in RNAP called the F bridge
\cite{bar-nahum_ratchet_2005}. The forward motion, the translocation step, is
on the other hand free to happen by Brownian motion. 

Nevertheless, pausing and backtracking do happen during elongation. As part of
the effort to understand why, it has been assumed that RNAP is in chemical
equilibrium between each nucleotide incorporation step
\cite{greive_thinking_2005}. This allows the reduction of RNAP movement into
the difference in free energy between a given state and all other possible
states, such as a paused state, a backtracked state, or a forward moving state
\cite{greive_thinking_2005}. While it is difficult to conclusively demonstrate
that the equilibrium assumption always holds true, it does facilitate the
formulation of the movement of RNAP as an equation:
\begin{equation}
	\Delta G_{\text{Total}} = \Delta G_{\text{RNA-DNA}} + \Delta
	G_{\text{DNA-DNA}} + \Delta G_{\text{RNAP}},
	\label{eq:rnap_energy_balance}
\end{equation}
This equation contains the terms currently believed to play a role in RNAP
movement, based on the interaction between RNAP and DNA. $\Delta
G_{\text{RNA-DNA}}$ is the free energy of the $\sim$ 8-9 bp RNA-DNA helix
inside RNAP, $\Delta G_{\text{DNA-DNA}}$ is the free energy of the $\sim$ 15 bp
DNA bubble that RNAP keeps open around it at all times, and $\Delta
G_{\text{RNAP}}$ is the non-specific free energy of other interactions between
RNAP, DNA, and RNA \cite{greive_thinking_2005}. By calculating these energies
for all possible reaction pathways for RNAP, one can presumably find the
pathway that is most energetically favorable, and thus map out the movement of
RNAP on DNA.

This conceptual model of RNAP movement seems orderly: one needs only to
calculate the change in free energy -- from available energy tables -- to find
out if RNAP will move forward, backtrack, pause, or terminate at any given
location on DNA. But one should consider whether historically the RNA-DNA and
DNA-DNA variables have been included in the model because they fit experimental
data or because their measured values have simply been available in the
literature for decades.

\subsubsection{Computational models of transcription elongation}
Several kinetic and thermodynamic models of transcription elongation have been
published \cite{tadigotla_thermodynamic_2006-1, bai_sequence-dependent_2004,
guajardo_model_1997}. What they have in common is that in some form they
incorporate the terms from \eqref{eq:rnap_energy_balance} and calculate the
$\Delta G_{\text{RNA-DNA}}$ and $\Delta G_{\text{DNA-DNA}}$ terms in the
equation \eqref{eq:rnap_energy_balance} from published energy tables
\cite{wu_temperature_2002, santalucia_thermodynamics_2004}. In the case of
Tadigotla et al. \cite{tadigotla_thermodynamic_2006-1} the free energy from the
nascent RNA secondary structure close to the RNA exit channel is used as well;
this is for the purpose of modeling the effect of strong RNA hairpins in
preventing backtracking of paused RNAP.

These models have had partial predictive power, but more work needs to be done
before a truly descriptive model of transcription is at hand. Since the models
do not fully live up to the theory, this means that there is more to RNAP
processivity than the thermodynamic variables in equation
\eqref{eq:rnap_energy_balance} on which these models are based. Indeed, it was
recently published that the 3\p dinucleotide of the nascent RNA has a strong
effect on translocation rates \cite{hein_rna_2011}, showing that there are
other factors at play than have been previously suspected.

\subsubsection{A computational model of transcription initiation}
The key differences between transcription initiation and transcription
elongation are that during initiation i) RNAP is bound to the promoter via
sigma, ii) the DNA-DNA bubble is scrunched and pulled into RNAP, and iii) the
RNA-DNA hybrid lacks its full length until the active site has reached +8/9.
The consequence of these difference is that RNAP has a reaction pathway during
initiation that does not exist during elongation: abortive initiation.

Based on equation \eqref{eq:rnap_energy_balance} and the above differences, a
model for transcription initiation was previously published
\cite{xue_kinetic_2008}. This model takes as input the ITS of a promoter and
encompasses the rate of transcription in the ITS, the abortion of initiation
with the accompanying release of nascent RNA, and the actual promoter escape
event. The authors manage to use the model to predict the maximum size of
abortive transcript and the abortive to productive ratio of the N25 promoter,
however they do not explain well the abortive probabilities. Some criticism of
this study is that only the data from the N25, N25anti and T7A1 promoters from
an older study were investigated, even though a more comprehensive dataset was
available \cite{hsu_initial_2006}. Further, the T7A1 promoter varies in the
core promoter sequence compared to the other two, which only vary in the ITS.
Variation in core promoter sequence sequence has been shown to affect rates of
abortive initiation \cite{vo_vitro_2003-1}, yet the authors have not taken this
variation into account in the model. The variation in core promoter sequence
between T7A1 and the other two promoters convolutes the overall interpretation
of the model output for the three promoters.

In conclusion, this study gave the first quantitative model for transcription
initiation, and manages to reproduce some of the experimentally measured
parameters. However, the model is not used predictively, making it difficult to
conclude how much of transcription initiation it truly captures. In the field
of transcription initiation, it remains to be published a quantitative model
with predictive power that gives new insight into the process itself.
