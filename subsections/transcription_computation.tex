%\addbibresource{/home/jorgsk/phdproject/bibtex/jorgsk.bib}

\subsubsection{Conceptual model for RNAP movement on DNA}
The movement of RNAP on DNA is thought to occur by a so called Brownian ratchet
mechanism. In this model, the backward movement of RNAP is structurally
prevented by the RNAP F bridge \cite{bar-nahum_ratchet_2005}, while the forward
motion is free to happen by Brownian motion.

Still, pausing and backtracking can happen. The cause for the backward motion
and other types of RNAP pausing has been explained by assuming that RNAP is at
chemical equilibrium between each nucleotide incorporation step. Then, the
reaction pathway chosen depends on the free energy difference between the
current state all other possible states, such as a paused state, a backtracked
state, a terminating state, and a forward moving state
\cite{greive_thinking_2005}. Two assumed factors where the free energy balance
is readily calculated is the RNA-DNA hybrid and the DNA-DNA bubble. The total
free energy difference has been formulated as

\begin{equation}
	\Delta G_{\text{Total}} = \Delta G_{\text{RNA-DNA}} + \Delta
	G_{\text{DNA-DNA}} + \Delta G_{\text{RNAP}},
	\label{eq:rnap_energy_balance}
\end{equation}
where $\Delta G_{\text{RNAP}}$ is the non-specific energy state of the RNAP at
a given position on DNA \cite{greive_thinking_2005}.

This conceptual model seems orderly, but one should consider whether
historically the RNA-DNA and DNA-DNA parameters have been included because they
fit the data or because they simply have been easy to measure compared to the
other interactions involved in transcription.

\subsubsection{Computational models of transcription elongation}
Since the first two terms of equation \eqref{eq:rnap_energy_balance} can be
calculated from published energy tables \cite{wu_temperature_2002, 
santalucia_thermodynamics_2004}, several kinetic and thermodynamic models
of transcription elongation have been published
\cite{tadigotla_thermodynamic_2006-1} \cite{bai_sequence-dependent_2004}
\cite{guajardo_model_1997}. What they have in common is that in some form they
incorporate the terms from \eqref{eq:rnap_energy_balance}. In the case of
Tadigotla et al. \cite{tadigotla_thermodynamic_2006-1} the free energy from the
nascent RNA secondary structure close to the RNA exit channel is used as
well. This is to aid the prediction of pause sites where a strong hairpin will
prevent backtracking while RNAP is paused.

These models have had partial predictive power, but more work needs to be done
before a truly descriptive model of transcription is at hand. The
descriptiveness of the current models make it clear that is more to to RNAP
processivity than the parameters in \eqref{eq:rnap_energy_balance}.  Indeed, it
was recently published that the 3' dinucleotide of the nascent RNA has a strong
effect on translocation rates \cite{hein_rna_2011}, showing that there are
other factors at play than have been previously suspected.

\subsubsection{A computational model of transcription initiation}
The difference between transcription initiation and transcription elongation is
i) that during initiation the RNA-DNA hybrid lacks its full length until +8/9,
ii) that the DNA-DNA bubble is scrunched and pulled into RNAP, and iii) that
RNAP is bound to $\sigma$ which again is bound to the promoter. The consequence
of these difference is that the TIC has a reaction pathway that doesn't exist
during elongation: abortive initiation.

Based on equation \eqref{eq:rnap_energy_balance} and the above differences, a
model for transcription initiation was previously published
\cite{xue_kinetic_2008}. This model attempts to cover explicitly both the
abortive process and the actual promoter escape event. Although these authors
manage to reproduce well the maximum size of abortive transcript and the
abortive to productive ration of the N25 promoter, they do not explain well the
abortive probabilities. Some criticism of this study is that it only the N25,
N25anti and T7A1 promoters were investigated, even though a larger dataset was
available \cite{hsu_initial_2006}. Further, the T7A1 promoter varies not only
in the ITS sequence compared to the other two, but also in the core promoter
sequence. Variation in the core promoter has been shown to affect rates of
abortive initiation \cite{vo_vitro_2003-1}, yet these authors have not taken
this variation into account.
