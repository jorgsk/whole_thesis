%\addbibresource{/home/jorgsk/phdproject/bibtex/jorgsk.bib}

\subsubsection{Conceptual model for RNAP movement on DNA}
The movement of RNAP on DNA is thought to occur by a so called Brownian ratchet
mechanism. In this model, the backward movement of RNAP is structurally
prevented by the RNAP F bridge \cite{bar-nahum_ratchet_2005}, while the forward
motion is free to happen by Brownian motion. Still, pausing and backtracking
does happen during elongation, and much effort has been devoted to understand
exactly how RNAP moved on DNA. To analyze RNAP movement, the current model
assumes that the movement of RNAP is dependent only on the free energy
difference between the current state and all other possible states, such as a
paused state, a backtracked state, a terminating state, and a forward moving
state \cite{greive_thinking_2005}. This assumption relies on RNAP being in
chemical equilibrium between each step; and while this may not always be the case,
it does facilitate the formulation of the movement of RNAP as an equation with
the free energies of the mentioned steps. This equation is given as
\cite{greive_thinking}
\begin{equation}
	\Delta G_{\text{Total}} = \Delta G_{\text{RNA-DNA}} + \Delta
	G_{\text{DNA-DNA}} + \Delta G_{\text{RNAP}},
	\label{eq:rnap_energy_balance}
\end{equation}

In this equation, $\Delta G_{\text{RNA-DNA}}$

Then, the reaction pathway chosen. Two assumed
factors where the free energy balance is readily calculated is the RNA-DNA
hybrid and the DNA-DNA bubble. The total free energy difference has been
formulated as
where $\Delta G_{\text{RNAP}}$ is the non-specific energy state of the RNAP at
a given position on DNA \cite{greive_thinking_2005}.

This conceptual model seems orderly, but one should consider whether
historically the RNA-DNA and DNA-DNA variables have been included because they
fit the data or because they simply have been easy to measure compared to the
other interactions involved in transcription.

\subsubsection{Computational models of transcription elongation}
Since the first two terms of equation \eqref{eq:rnap_energy_balance} can be
calculated from published energy tables \cite{wu_temperature_2002, 
santalucia_thermodynamics_2004}, several kinetic and thermodynamic models
of transcription elongation have been published
\cite{tadigotla_thermodynamic_2006-1, bai_sequence-dependent_2004,
guajardo_model_1997}. What they have in common is that in some form they
incorporate the terms from \eqref{eq:rnap_energy_balance}. In the case of
Tadigotla et al. \cite{tadigotla_thermodynamic_2006-1} the free energy from the
nascent RNA secondary structure close to the RNA exit channel is used as well.
This is to aid the prediction of pause sites where a strong hairpin will
prevent backtracking while RNAP is paused.

These models have had partial predictive power, but more work needs to be done
before a truly descriptive model of transcription is at hand. The state of the
current models make it clear that is more to to RNAP processivity than the
thermodynamic variables in equation \eqref{eq:rnap_energy_balance} on which
these models are based. Indeed, it was recently published that the 3\p
dinucleotide of the nascent RNA has a strong effect on translocation rates
\cite{hein_rna_2011}, showing that there are other factors at play than have
been previously suspected.

\subsubsection{A computational model of transcription initiation}
The key differences between transcription initiation and transcription
elongation are that i) during initiation the RNA-DNA hybrid lacks its full
length until +8/9, ii) the DNA-DNA bubble is scrunched and pulled into RNAP,
and iii) RNAP is bound to sigma which again is bound to the promoter. The
consequence of these difference is that RNAP has a reaction pathway during
initiation that does not exist during elongation: abortive initiation.

Based on equation \eqref{eq:rnap_energy_balance} and the above differences, a
model for transcription initiation was previously published
\cite{xue_kinetic_2008}. This model attempts to cover explicitly both the
abortive process and the actual promoter escape event. Although the authors
manage to reproduce well the maximum size of abortive transcript and the
abortive to productive ration of the N25 promoter, they do not explain well the
abortive probabilities. Some criticism of this study is that it only the N25,
N25anti and T7A1 promoters were investigated, even though a larger dataset was
available \cite{hsu_initial_2006}. Further, the T7A1 promoter varies not only
in the ITS sequence compared to the other two, but also in the core promoter
sequence. Variation in the core promoter has been shown to affect rates of
abortive initiation \cite{vo_vitro_2003-1}, yet the authors have not taken
this variation into account.
