%\addbibresource{/home/jorgsk/phdproject/bibtex/jorgsk.bib}
The three topics that have been investigated -- sequence dependent abortive
initiation, sequence dependent translation initiation, and sequence dependent
cleavage and polyadenylation -- all center around RNA. It has been rewarding to
learn about RNA from these different angles, to realize that there is so much
more to this molecule than simply being an obligatory intermediate between DNA
and protein.

One can ask what field of research this thesis falls in. The work of modeling
the movement of RNA polymerase during transcription initiation falls in the
field of mathematical biology, whereas the work with translation initiation and
polyadenylation of RNA falls in the field of bioinformatics. Broadly, then, the
work falls in the field of computation biology, which is a field of biological
research where the computer algorithms are used as an extension of the human
mind because the computation that must be done involves too many steps for the
human mind to perform alone. Much has been said about the merger of mathematics
and biology, recently popularized under the brand systems biology
\cite{kitano_systems_2002}. Some have had the hope that systems biology would
deliver different ``theories of biology`` in the same way that there are
theories of physics, like Newton's laws of motion or the standard model of
particle physics. There has even been published a book called ''The philosophy
of systems biology``, as if there is something fundamentally different about
this approach to studying biololgy. In the end, computational and systems
biology is using mathematical and statistical methods to study biological data.

In the end, however, the point of any work with biological data is to obtain
more information about the biological system from which that data came. To add
knowledge to a field in biology means that the researcher must know that field
already. What I have come to realize during my PhD is that a computational
biologist, if he or she wants to be successful, must be first and foremost be a
bioligist, second of all a computer scientist, physicist, or mathematician.
Only then will she be able to ask the interesting questions out there. If the
researcher remains primarily a physicist, this person will ask questions that
are more interesting to physics than to biology, and this person will forever
be relying on collaboration with biologist to obtain relevant research
questions. A computational biologist must anyway collaborate with biologist in
order to obtain experimental data. But if the research question itself came
from the biologist, then that biologist will be the main or senior author of
the work that comes out. Does that mean that a computational biologist must
work twice as hard as a biologist in science? Not really, since a computational
biolgist doesn't have to learn how to actually do biological experiments -- he
must understand how the experiments are performed, and what are the implication
of the experimental steps on the output data -- but he does not have to spend
time and effort in learning the techniques and performing the experiments in
the lab.

Given this limitation, there are many biologically interesting results that a
computational biologist can never discover. As an example of 



Another thing that has been rewarding 

different methods and lie in slightly different fields of molecular genetics.

3) Three topics united by RNA.

What do you want to say?

1) That systems biology and bioinformatics often does not work directly with
experimental data, effectively only doing part I of the great cycle of
scientific research

2) That it is not enough to just build a model for something; that's a first
step. If that step is not taken into the predictive step, very little new
knowledge has come out of the model, because we cannot know if it captures the
system's dynamics.

