%\addbibresource{/home/jorgsk/phdproject/bibtex/jorgsk.bib}

\subsubsection{Summary of the paper}
The problem to be solved was to increase the expression of some gene. To do
that, they ordered a codon optimized version of the same gene. However,
expression was nearly 10 times worse with the codon optimized version than the
wild type (wt), and I was employed to find out why.

To add to the mystery, when expressing the gene with a His-tag, the wt and co
(codon optimized) versions performed similarly.

Further, when exchanging 5UTR sequence with the 5UTR for T7, the expression of
both wt and co are high and identical (2000).

\subsubsection{My contribution}

My contribution was to identify a strong consistent hairpin in the codon
optimized gene. This hairpin formed within the ribosome binding site (RBS), +12
relative to translation start. By adding the His-tag, this strong hairpin would
effectively be pushed away from the RBS. Based on this, I suggested trying out
3 versions of the His-CO construct, where the His-tag is gradually increased in
length. This showed that XXX. Rahmi?

I also proposed that the difference between T7 (2000) and His-tag (300/650) is
due to a poor match whith the Shine-Dalgarno sequence in the wt compared to a
great SD-match for T7.

The strong hairpin of the codon optimized gene is independent of the 3UTR
region. The hairpin appears when folding the gene alone. This shows that what
is codon optimal is not always expression optimal.
