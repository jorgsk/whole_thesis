%\addbibresource{/home/jorgsk/phdproject/bibtex/jorgsk.bib}

\subsubsection{Summary of the paper}
The study by Swik et. al focuses on the expression of the xylS gene. The XylS
gene codes for the XylS protein, which is a transcriptional activator and part
of the Pm/xylS expression system. The Pm/XylS expression system consists of the
XylS protein, the Pm promoter, and m-toulic acid as the inducer. When passively
abosorbed m-toulic acid activates XylS, XylS binds to the Pm promoter and
activates it by recruiting RNAP \cite{gallegos_arac/xyls_1997,
inouye_expression_1987}.

To increase expression of the \textit{Pseudomonas putana} native xylS gene in
\textit{E. coli}, an \textit{E. coli} codon optimized version of the xylS gene
was obtained. The codon optimized version was called 'syn' as opposed to the
wild type 'wt'. Instead of increasing expression, 'syn' had 8-fold lower
expression than 'wt'. This difference in expression between 'wt' and 'syn'
persisted when using the T7 5\p UTR instead of the wild type 5\p UTR. The T7
5\p UTR increased expression for both gene versions, but the fold-difference in
expression remained the same. However, when both gene versions were fitted with
a His-tag as a 5\p fusion parter, their expression levels equalized at a high
level (depending on Rahmi's results). This indicates that the His-tag confers a
mechanism which completely (partially) mitigates the expression-reducing effect
of the 'syn' gene variant. The His-tag alone also equalized expression when
using the xylS wild type 3\p UTR, albeit with lower expression levels than for
the T7 3\p UTR.

\subsubsection{My contribution}
My first contribution was to identify a strong and consistent hairpin +12
downstream the ATG start codon in the codon optimized gene'. This strong
hairpin did not form in the wild type gene, and thus the hairpin is a result of
the combination of optimized codons. Strong hairpins proximal to ATG have
previously been shown to reduce gene expression \cite{seo_quantitative_2009},
likely by interfering during translation initiation.

I proposed three different tests to investigate whether the ATG-proximal
hairpin was behind the reduction in expression from WT to syn. The first test
was to weaken the hairpin with synonymous codon changes, and see if the
constructs with weakened hairpin have increased expression. The original
hairpin was XXX in strength. I proposed two alternative 'syn' versions, 'syn1'
with a hairpin of YYY, and 'syn2' with a hairpin of 'XXX'. The expression of
these were Y and Z, which shows that the hairpin strength was a partial cause
for the low expression of the 'syn' construct.

The second test was designed to evaluate the effect of the hairpin on
translation initiation from a different angle. Since 'syn' was lowly expressed
with the T7 5\p UTR alone, but highly expressed with the His-tag, I proposed
that the His-tag increased expression by removing the 'syn' hairpin from the
translation initiation region. When used as a 5\p fusion, the ATG of the His
tag becomes the translation start site, effectively putting 64 nucleotides
between ATG and the hairpin. My proposal was to gradually build the His-tag
onto the T7+'syn' combination, which should gradually increase expression was
the hairpin would be moved further and further away from the start codon. The
result of the experiment was that expression did increase as the distance
between the ATG and hairpin was increased, although even with 24 nucleotides
distance the expression was not equivalent to the full His-tag. This indicates
the the function of this His-tag is only partially because the separation of
the translation start site and the strong secondary structure formed by the
'syn' sequence.

In summary, the tests I proposed showed that the hairpin formed by 'syn' in the
5\p region of the gene was partially responsible for the decreased expression
of 'syn' relative to 'wt'. However, there are also clear downstream effects of
'syn' that are not related to the hairpin. This shows that codon optimization
is not a universal technique to increase expression, since the new sequence
formed by optimal codons may cause unwanted effects during gene expression. 

